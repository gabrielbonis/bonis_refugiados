\chapterspecial{Quinto diálogo -- Teoria da psicanálise social}{}{}
 

No último diálogo Jacob Pinheiro Golberg sugere que a dramaturgia
psicológica que é investigada pela psicologia imagética, surge num
contexto social, que deve ser investigada por uma psicanálise social.

Segundo Goldberg, a história dos povos alterna um dualismo: a paranoia,
que é aliviada por um movimento de metanoia. Cada sujeito vivencia sua
história através de diversas narrativas paralelas ou entremeadas, e a
todo momento está optando por uma posição paranoica ou metanoica. Em
cada fragmento de vida, a pessoa opta por estar numa das duas posições,
de acordo com a facilidade de integração, socialização ou mesmo da
própria sobrevivência dessa pessoa dentro do grupo maior no qual está
inserido.

Porém, quando a aflição se torna insuportável por causa de uma posição
assumida em determinado fragmento de vida ou numa determinada narrativa,
num ato de sanidade a pessoa procura a psicanálise. Para Goldberg, essa
é a forma de agir da pessoa em relação não apenas ao desconforto
psíquico, mas também o desconforto físico, a doença, quando a pessoa
procura tratamento.

Os limites para a paranoia e a metanoia são a radicalidade e a
transigência. Goldberg acredita que estes últimos conceitos atuam
ampliando ou diminuindo a intensidade da aflição do ser humano. Um
paranoico intransigente sofre. Um metanoico radical, sofre. Em suas
palavras, ou você vai para o hospício, ou vai para a cadeia. Neste
sentido, quanto mais paranoico intransigente, mais sádico é o
comportamento. Por outro lado, quando mais metanoico transigente, mais
masoquista é o comportamento.

Para Goldberg, a psicanálise é uma tentativa que o individuo se permite
fazer, para tentar compreender se ele é um sujeito que não está
inserido no ajuste harmônico do mundo, ou ao contrário, se o mundo
apresenta para ele um caráter de opressão, de brutalidade e violência ao
qual ele precisa reagir.

Neste sentido, a psicanálise precisa entender o sujeito dentro da
dramaturgia psicológica da sociedade em que está inserido. Não se pode
manter padrões e hábitos eurocêntricos na psicanálise de outros países e
outras culturas. A~psicanálise ortodoxa é um dos instrumentos para
adoentar a sociedade.

A pessoa normal oscila o tempo todo no seu comportamento. Por outro
lado, o banqueiro que descansa numa praia francesa enquanto crianças
inocentes sofrem por causa da política de juros, este é doente mental.

A psicanálise de Goldberg é aquela que permite o desvelamento das razões
da aflição do ser humano. O~trabalho psicológico tem que ser uma
anunciação seguida por um processo de revelação. Um processo através do
qual a pessoa se revela para ela mesmo, do jeito que quiser e puder. Mas
há também uma questão moral. A~pessoa pode perfeitamente fazer a opção
de saber ou não saber o motivo de sua aflição.

O sujeito tem que expurgar de si, o alienado no qual a sociedade o
transformou. A~cura é a liberdade. É~a pessoa viver, simplesmente
segundo o seu próprio destino escolhido. O~destino deve ser escolhido. A~liberdade é a maturidade, individual e dos povos.

Goldberg acredita que a pessoa pode inclusive escolher viver com suas
dores psíquicas, se escolher este caminho. Para ele, a consciência
social não torna a pessoa imune ao sofrimento. O~sofrimento depende da
sua natureza, depende da realidade individual. Desde que o sujeito
consiga manter os seus núcleos de dignidade, os seus núcleos pessoais
preservados, a dor é suportável.

Essa dor não tem características de alienação e alheamento, ela não é
mais uma dor neurótica. Ela é uma dor de existir. É~possível lidar com
ela de maneira razoavelmente suportável. Ela deixa de ser uma doença
para ser simplesmente a função de uma diferença individual de escolha da
pessoa, em relação aquilo que é a realidade do mundo. Por vezes pode até
ser um estado de conforto, mesmo se esta opção custar um preço muito
alto de renúncia, de solidão, de incompreensão, pois a independência tem
esse preço.

E Goldberg faz uma denúncia, que considera obrigatória: quando um
terapeuta trabalha com a opinião das pessoas, com a posição das pessoas,
com a maneira das pessoas pensarem e se conduzirem, ele não pode se
postar como um salvador. Há salvadores que em nome da esquerda, da
direita ou qualquer posição, tem a pretensão de serem intérpretes, e tem
a intenção e o desejo (seja político, ideológico, jurídico, religioso de
qualquer grupo), de serem proprietários da verdade. Mas isso não é
possível, pois não existem certezas pacíficas.

\begin{center}\asterisc{}\end{center}

  \abrefala

\textbf{Renato} \textbf{Bulcão}: Existe uma psicologia brasileira? A sua
proposta do que seria uma contribuição à psicologia mundial ou global,
acaba sendo também brasileira. Porque é uma contribuição brasileira.

Pois me parece que a sua ideia é a seguinte: Na medida em que os povos
trabalham com um movimento de paranoia e se aliviam num movimento de
metanoia, automaticamente cada sujeito trabalha a sua vida fragmentada
e as suas diversas narrativas paralelas ou entremeadas, de tal maneira a
estar, ou do lado da paranoia; ou do lado da metanoia; conforme a
facilidade de integração, socialização e mesmo a sobrevivência dessa
pessoa dentro do grupo maior.

Nesse sentido, não se pode falar em doença mental. Porque seja psicose,
neurose ou o que for nós estamos o tempo todo percebendo que existe no
fragmento de vida que uma pessoa está vivendo, uma melhor ou pior
adequação do individuo à socialização dele no grupo, dentro do contexto
da narrativa daquela fragmentação que ele está vivendo.

 

Então se ele vive uma história com a mãe, é uma coisa. Com o patrão é
outra. Com a mulher é outra. Com o filho é outra e assim sucessivamente.
A~doença acaba sendo justamente a impossibilidade do individuo suportar
o tamanho da defasagem que ele tem num determinado fragmento, e isso
causa nele extrema aflição, que é o termo que você usa. Você não usa
angustia, você usa aflição, extrema aflição.

 

Justamente porque a pessoa está aflita com aquele viés da vida dele num
determinado momento, ou por uma determinada causa, que pode ter origem
na paranoia ou na metanoia, a pessoa então, num ato de sanidade, procura
a psicanálise no sentido de tentar primeiro conversar a respeito daquele
juízo de realidade que ela está tendo. Segundo, verificar se o juízo de
valor que ela está tendo sobre aquele juízo de realidade, faz sentido ou
não faz sentido?

 

\textbf{Jacob Pinheiro Goldberg}: Você colocou a questão que tem sido a
pergunta enigmática em relação à doença mental. Mas eu talvez ousasse
dizer, que em parte é a doença quase que em geral. Nesse aspecto, eu
acho que a quintessência dessa noção eu apresentei no trabalho ``Eve
Will be God'' na Faculdade de Medicina de Londres, que é a questão da
doença psicossomática da mulher.

 

Então do mesmo jeito que antes eu disse que eu era negro, nesse capitulo
eu diria que quem fala em nome da mulher sou eu. E~eu me lembro que eu
participei de um congresso feminista em São Paulo, e uma senadora disse
que eu era muito radical no meu feminismo. É~muito curioso, inclusive
porque essa radicalidade tem sido atribuída, a todas as minhas posições.

 

Eu acredito que este é um dos elementos fundamentais que a gente deveria
abordar aqui, que é a questão da transigência e de radicalidade,
acompanhando os conceitos de metanoia e paranoia. O~sistema de pressão
anti"-individual, ele é radical. Ou você se achata, ou você vai para o
hospício, ou vai para a cadeia. No limite é isso. Existem as baldeações
e as estações intermediárias, mas esses são os limites.

 

Enquanto ele não tem essa consciência, que é um processo de reflexão que
a análise pode proporcionar, mas dificilmente proporciona, porque suas
raízes filosóficas estão viciadas. Essa foi uma preocupação que eu tive
o tempo todo, de desenraizar. Não era possível você pegar aquela árvore
e plantar em outro solo. Desta maneira ela não poderia ser eurocêntrica.

 

Ela não poderia ser judaico --- germânica no Brasil. É~nisso que eu
insisto muito. Eu creio inclusive que eu finquei o grande momento, num
programa de televisão que foi ``O Jogo da Verdade''. Eu contestei e
acredito que era preciso introduzir o sincretismo na psicanálise, do
mesmo jeito que existe no Brasil o sincretismo na religião.

 

Eu falei na herança da magia e na herança da religião. Cada vez mais eu
estou convicto disso. Para que ela possa realmente servir para o
individuo e para a sociedade no Brasil, ela tem que se abrasileirar, ela
tem que ser brasileira.

 

\textbf{R}: Isso significa que é independente do Brasil, ela tem que se
uruguaizar, se argentinizar, se chilenizar?

 

\textbf{G}: Exatamente, No Brasil ela tem que se avacalhar.

 

\textbf{R}: Por que avacalhar?

 

\textbf{G}: Porque o brasileiro é avacalhado, porque o Brasil é
avacalhado. Porque nós somos avacalhados. Porque a organização é
fascista, a ordem é fascista, o totalismo é anti"-humano. O~Brasil, na
minha opinião, por uma série de circunstâncias que não caberia discutir,
é uma sementeira de possibilidades de modificação dessa condição
anti"-humana. Nesse aspecto, é um dos pedaços do território da terra
prometida da utopia de Kandire.

 

Kandire é aqui também. Não é só aqui. Kandire também é um solo
palmilhado por Fernando Pessoa, quando diz que conhecer suas emoções e
falar delas, é enfrentar o susto. O índio brasileiro fala que qualquer
problema mental é oriundo da doença do susto. E~o que é o susto? É a
surpresa diante da Revelação. O~que nós estamos discutindo o tempo todo
aqui, é, revelar é desmitificar.

 

Revelar é denunciar, revelar é nascer, revelar é dizer: Essa dor que eu
sinto aqui no braço é porque em um certo instante, quiseram paralisar
esse meu braço. Então eu coloquei a dor no meu corpo. É~preciso
exorcizar a dor do corpo. É~preciso tirar a dor desse corpo.

 

O dia em que a nomenclatura se transformar e ao invés de se dizer: ``O
sujeito está com câncer'', se disser, ``O sujeito está sendo
apodrecido''; muda a leitura! No dia em que o diagnóstico não for: ``É
doença cardíaca!'' --- mas: ``É doença do sentimento, é doença da emoção
afetiva, é doença do seu lado bondoso.''. ``O sujeito teve um enfarte
cardíaco, e ele não aguentou mais, não suportou mais''.

Que ao menos a gente dê oportunidade para que ele lute, para que ele
possa reagir. O~grande problema é manietar o sujeito antes. Você
entorpece ele com alienação, com mistificação e de certa maneira com
remédio.

 

Se o remédio for dado para fortalecer, no sentido de lutar, é válido. No
sentido de anestesiar, é matá"-lo.

 

\textbf{R}: Eu entendi a escala entre de um lado, o dominante e o
sadismo; e de outro lado, uma escala de se vergar a esse sadismo
masoquisticamente, até o limite do hospício e da cadeia.

 

De outro lado, existe a ideia de que todo lugar é ou pode ser o paraíso
prometido, e aqui (o Brasil) teria a sua vocação de paraíso prometido.
Eu estou insistindo muito na ideia primeira da fragmentação do
sujeito, porque a sua primeira colocação, a sua primeira visão que
impede que o individuo seja visto como 100 \% neurótico ou 100 \%
psicótico ou 100 \% qualquer coisa, ninguém é doente 100\%.

 

Aparentemente existe uma busca sua em localizar em que fragmento, ou
seja, em que momento, a pessoa não está aguentando mais, ou tolerando
mais, e eventualmente está precisando se vergar ou eventualmente está
precisando reagir a essa situação que ele está vivendo.

 

Então, no fundo, e aqui é minha provocação, a psicanálise nada mais é
que do que um instrumento de reajuste social?

 

A psicanálise , vista de paletó e gravata, com charuto ou cachimbo, com
a consulta ao preço que é, nas circunstancias de pretensão e arrogância
inumana com que se afirma, é um dos instrumentos mais poderosos para
adoentar a sociedade.

 

Do mesmo jeito que o tratamento pavloviano era feito pela ditadura
stalinista, (a ditadura anticomunista de Stalin), e do mesmo jeito que a
psicanálise era proibida na Alemanha nazista e substituída pela religião
do opróbrio.

 

Tentando colocar melhor a resposta, o sujeito que tem na nomenclatura
convencional, na estrutura convencional, a oscilação que é bipolar, esse
sujeito é normal. O~anormal é o banqueiro que fica contando dinheiro
numa praia francesa sabendo quantas pessoas, quantas crianças, quanta
gente inocente está sofrendo profundamente, para que ele viva uma vida
completamente esquizotímica. Esse é maluquinho. Este é o doente mental.

 

O doente mental é o dono do convênio médico. Esse é um doente mental,
aliás, perigosíssimo. Porque ele é o doente mental na fronteira do
crime. É~perverso, procurando vantagem. O~que pode ser mais doente
mental do que um dono de convênio médico no Brasil, dono de mais ou
menos 300 milhões de reais? Ele vive uma vida completamente fora da
realidade. Essa é uma doença mental. Agora, o doente que foi ao
hospital, que pagou o plano pra esse sujeito, e que está com tique
nervoso no rosto, esse é doente (para o sistema).

 

Aquele que fica o dia inteiro na academia fazendo flexões com milhões de
reais no banco é normal? Não! Não é, na minha opinião, definitivamente
não é. Então é preciso colocar de ponta"-cabeça o processo. É~preciso
explicar para esse sujeito que tem um tique, que esta é só uma reação do
organismo para que ele não fique de uma vez psicótico. Que o corpo dele
está num processo de reação. Que o corpo dele está dando o sinal amarelo
e que daqui a pouco vai virar sinal vermelho. Quando ele fica psicótico
ele morre enquanto identidade, porque ele não está suportando as
pressões.

 

\textbf{R}: Nesse sentido então, está confirmado, é uma questão de
reajuste social.

 

\textbf{G}: A psicanálise que tiver a intenção de dizer para esse
sujeito que oscila o humor várias vezes por dia, entre outras coisas,
porque a mulher dele está praticando adultério com o primo dele…
Se você tenta fazer um trabalho que filosófica e emocionalmente sinaliza
em direção a que ele se conforme com essa realidade, porque faz parte do
contexto de vida e blablabla, é um instrumento que serve ao ajuste ou
reajuste social da dominância.

 

\textbf{R}: Então isso significa na prática, que a psicanálise na qual o
senhor acredita, tem em princípio duas grandes vertentes. A~primeira
vertente é a ressignificação dos processos de ajuste social, a partir de
um viés marxista. Essa é a primeira vertente.

 

A segunda vertente seria a possibilidade de você permitir ao individuo
dar vazão à sua aflição imediata, na medida em que você explica
diretamente para ele, sem rodeios, mas numa conversa franca e aberta,
que aquilo que ele está sentindo, que o sofrimento e a aflição são
facilmente identificáveis, que é justamente um fragmento da vida dele do
qual ele não consegue dar conta.

 

\textbf{G}: Eu só quero deixar claro que não pode ser um processo
político didático. Quer dizer, não pode ter esse caráter através do qual
se discuta uma questão muito mais política e filosófica, do que um
trabalho psicológico.

 

O trabalho psicológico tem que ser uma anunciação seguida por um
processo de revelação. Um processo através do qual ele se revela para
ele mesmo, do jeito que ele quiser. Porque é uma questão também moral.
Ele pode perfeitamente fazer a opção de saber ou não saber.

 

``Eu quero ser banqueiro! Já que eu estou sendo atingido… Eu
quero ser eu! O dono do plano de convênio. Me interessa muito mais, é
esse o gozo que eu pretendo…''.

 

\textbf{R}: Quer dizer que o gozo dele pode ser perverso e está tudo
certo?

 

\textbf{G}: Pode ser perverso. Aí é com ele, sua consciência e
responsabilidade. Outra coisa, que eu também gostaria de consignar aqui
para não existir dúvida, que talvez tenha ficado uma dúvida que eu mesmo
tenha provocado. Eu não tenho uma leitura marxista nem de vida, nem de
mundo, nem de nada.

 

O próprio Marx tinha dentro da casa dele, no comportamento pessoal dele,
uma contradição que eu acredito que contamina inclusive a obra dele. Eu
acho que é impossível distinguir o homem da sua obra. A~relação dele com
a mulher, com a família, já começa daí e também vai terminar aonde?
Quando o marxismo foi aplicado, mesmo que de maneira mórbida, patológica
e doentia, deu num desastre.

 

Então eu tenho muito receio de me enquadrar em qualquer sistema fechado
de pensamento. Prefiro muito mais a posição, a posição anarquista,
aquela que se nega a qualquer cerebração rígida. Não tem cerebração
rígida. É~mesmo libertária.

 

\textbf{R}: Isso significa que como não tem celebração rígida, a pessoa
então pode celebrar aquilo que ela escolher, e eventualmente torná"-la
feliz?

 

\textbf{G}: É curioso porque eu falei cerebração e você entendeu
celebração. Eu acho interessante que tenha havido aqui esse
desvirtuamento de audição. Eu acho muito interessante, porque de alguma
maneira eu acredito que a celebração ela está ligada a descerebração.
Quando você coloca no cérebro, você limita. Não é no cérebro, daí a
metanoia, daí a diabolização do carnaval, daí a capoeira. A~estética da
capoeira, eu acho muito mais ligada á forma do psicodrama, a forma da
maneira pedagógica que eu entendo do teatro e da psicanálise, que é uma
espécie de aprendizado da desaprendizagem.

 

\textbf{R}: Explica melhor esse conceito do aprendizado da
desaprendizagem? Porque se não vai ficar uma coisa extremamente
desestruturante a título de terapia.

 

\textbf{G}: Um dos grandes pensadores do século passado se deu o
codinome de Janusz Korczak alterando o nome de origem judaica para o
nome polonês. Um dos livros mais importantes que ele escreveu foi
``Quando eu voltar a ser criança''. Antes da guerra ele propôs que um
dia por ano, as crianças da Polônia elegessem a presidência da república
e todo sistema de governo e assim era feito, por incrível que pareça,
antes da Segunda Guerra Mundial.

 

Imagina só o status de conflito que havia na Polônia naquela época,
entre já o fascismo e o progressismo. Então um dia por ano, uma criança
assumia a presidência da Polônia. E ``Quando eu voltar a ser criança'' é
um apelo no sentido da descerebração. A~filosofia do Korczak era
recuperar a ideia infantil da ingenuidade, da esperança, do amor, da fé
e tudo isso vai na contramão da organização civilizatória que se
pretende adulta, madura e saudável psicologicamente.

 

Na verdade a cura da doença do sofrimento e da aflição está muito mais
perto do festival do que do enterro. O~enterro é nobre. Não existe na
minha opinião, nada mais comovedor no sentido nobilitante e do épico do
que a música do silêncio.

 

Quando eu servia no 4º Regimento da Infantaria, (sempre lembrando que
foi bem antes de 1964), eu comandava a Primeira Companhia de Fuzileiros,
a noite na hora do toque de recolher, não tem nada mais comovedor e
fascista do que aquele toque. Vão dormir, vão dormir, porque amanhã será
a guerra. Vão cuidar do corpo e da saúde, vão ficar fortinhos e
bonitinhos para amanhã morrer na guerra. É~lindo aquilo, é estético
aquilo, é a estética do fascismo. É~a estética da saúde das academias,
do homem bombado, do Deus grego. O~contrário disso é a desorganização, é
a descerebração, é o moleque correndo num campinho de futebol sem
regras. É~a liberdade digna.

 

\textbf{R}: Ficou claro que não existe cura. Não tem do que ser curado.

 

\textbf{G}: Você tem que ser curado da doença que foi inoculada. Você
tem que expurgar o alienado para o qual você foi cooptado. É~a
liberdade, é o senso de liberdade. A~cura é a liberdade. É~você viver
simplesmente segundo o seu próprio destino escolhido. O~destino que você
escolhe. A~liberdade é a maturidade individual e dos povos.

 

\textbf{R}: Então basicamente o sofrimento e a aflição que a pessoa têm
é a incapacidade dela perceber que aquela dor que está sendo embutida
nela socialmente, no sentido que pode ser pela família, pode ser pela
religião, socialmente nesse sentido. A~dor não deve impedir a pessoa de
escolher, usando a sua palavra, o seu próprio destino…

 

\textbf{G}: Pagando eventualmente e inclusive as dores por essa escolha.
Mas aí são as suas dores, não é mais a sua doença.

 

Não é uma ingenuidade de presumir que na medida em que você tem
consciência absolutamente social, que você estará imune a qualquer
sofrimento. O~sofrimento depende da natureza, depende da realidade. A~gente envelhece, os tecidos envelhecem, existe uma deterioração
progressiva. Mais do que já ficou evidente e óbvio, desde que o
sujeito consiga manter os seus núcleos de dignidade, os seus núcleos
pessoais preservados, essa dor é suportável.

 

Essa dor não tem características de alienação e alheamento, ela não é
mais uma dor neurótica. Ela é uma dor de existir. Essa é minha
experiência pessoal e a experiência que eu vejo nos meus clientes, dor
essa inclusive que é possível de levar e de lidar com ela de maneira
razoavelmente suportável.

 

\textbf{R}: Ela deixa de ser uma dor para ser simplesmente a função de
uma diferença individual de escolha da pessoa, em relação aquilo que é a
realidade do mundo.

 

\textbf{G}: E de um estado de conforto. A~mais das vezes é muito comum
que essas opções que a gente está discutindo aqui, custam um preço muito
alto de renúncia, de solidão, de incompreensão, a independência tem esse
preço.

 

\textbf{R}: Você nesse sentido de uma maneira se liga menos aos outros.

 

\textbf{G}: Exatamente. Uma coisa que eu gostaria de complementar
terminando, é dizer o seguinte: Alguns anos atrás eu insistia muito na
importância do aqui e do agora. Do individuo viver o seu presente.

 

Se eu vivi uma transformação, e eu creio que tenha vivido, no pensamento
e na ação, foi a desimportância do aqui do agora. É~cada vez mais esse
sentido de Transcendência e de Eternidade. Esse sentido de muito a
maior, que eu acredito que dê para o ser humano, um mínimo de recursos
para reagir, diante da brutalidade, da infâmia.

 

Para terminar com aquilo que eu acho que é uma denúncia, que sempre tem
que ser feita e é obrigatória: quando você trabalha com a opinião das
pessoas, com a posição das pessoas, com a maneira das pessoas pensarem e
se conduzirem, muito cuidado com os salvacionistas.

 

Como aqueles que em nome de esquerda, direita, centro, meio, alto, baixo
ou qualquer posição, tenham a pretensão de ser intérprete, e tenham a
intenção, o desejo, a prerrogativa seja qual for: política, ideológica,
jurídica, religiosa de qualquer grupo, de serem proprietários de
qualquer verdade.

Não da Verdade --- essa está muito distante do que se possa sequer
imaginar. A~certeza é uma terra indomável.

Só viver, é para todos. Todos, menos um.%\footnote{Esta entrevista está disponível na íntegra em~goo.gl/a9wdwQ.} 

\fechafala 
