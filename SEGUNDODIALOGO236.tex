\chapterspecial{Segundo diálogo}{O carisma do homem comum}{}
 

 

Neste segundo diálogo, Jacob Pinheiro Goldberg afirma sua descrença nas
categorias predeterminadas de conceitos de distúrbios mentais. O~ser
humano não pode ser reduzido, em nenhuma hipótese, em nenhuma
circunstância, a moldes e modelos referenciais. Só quem se beneficia com
isto é a ``indústria da loucura'', liderada pela indústria farmacêutica.

Mas ele acredita que na hipótese, em que a própria ideia de doença é
calibrada segundo o nível de insatisfação ou de sofrimento da pessoa. A~própria pessoa que poderia se auto"-classificar, de acordo com seu nível
de insatisfação com a vida e consigo mesmo.

Por outro lado, qual a hipótese que permite que um sujeito normal,
estupre e mate uma criança e mata? Está claro que esta pessoa está longe
de qualquer normalidade, e não deveria ser julgada pelo direito, que
pressupõe a lei para pessoas normais. Portanto, existem coisas na
cultura que não podem ser explicados pela lógica.

Uma das manifestações que parecem carecer de lógica é a do carisma. Como
explicar o fascínio das massas por notórios homens públicos? Esses
homens, sejam eles políticos ou desportistas, não tem medo de se
apropriarem de signos comuns da cultura, e de resignificarem esses
signos de uma forma que façam sentido para o homem comum. Por outro
lado, fazem questão de demonstrarem que permanecem homens comuns. Isso
cria a possibilidade de projeção das massas nessa pessoa, porque o
público entende o que ele está falando e não sente medo deste discurso.

Segundo Goldberg, o olhar desse tipo de indivíduo é o olhar que leva em
conta o outro. Ninguém é desimportante para o carismático. Mas ele não
se limita a uma pessoa e seu olhar atravessa as pessoas na multidão,
olhando para todos. Para ele, um visionário opera um conceito muito
próximo do carismático.

Através da política, mais do que talvez qualquer outra atividade humana,
surge a real possibilidade de mudança da sociedade. Neste sentido o
carisma pode auxiliar o interesse pessoal e narcísico do indivíduo, e
surge o corrupto, o canalha, o medíocre, o infame. Estes são os
traidores da esperança. A~esperança é aquilo que o ser humano tem de
mais deslumbrante, que é a crença no outro. Quando você trai a crença
que o outro deposita em você, é pérfido.

Historicamente a esquerda é a tentativa daqueles que acreditam na
possibilidade de convivência e cooperação com o outro. A~direita é
aquela que não acredita no outro. É~de direita quem se imagina melhor
que o outro e poderia hierarquizar qualquer processo desde que liderado
por ele mesmo. E~isso poderia trazer benefícios pessoais.

Então por que a maior parte das pessoas, se inclina mais à direita mesmo
sabendo que a escolha da esquerda é que iria indiretamente
beneficiá"-las? Por que a esquerda não existe de forma homogênea. Existem
muitas esquerdas e uma direita. E~a direita pretende deter o rumo da
história, dos processos humanos. A~direita se esforça para que a
sociedade permaneça infantilizada, enquanto uma casta detém o poder.
Seja o poder da cultura, militar, social, financeiro ou religioso em
todas as atitudes e em todos os níveis.

 

Na esquerda existem várias esquerdas. E~também existe na esquerda,
matizes que na realidade são profundamente de direita. Não podemos
esquecer que Hitler chefiava um partido que se chamava Partido Nacional
dos Trabalhadores Alemães. Hitler alegava que era nacional --
socialista.

A posição de Goldberg é anarco"-socialista. Apoia a liberdade individual
extrema: Ou é livre, ou não é. Segundo ele, o ser humano não tem outra
alternativa pois nasceu e tende para a liberdade. Já o socialismo seria
a única forma de distribuição da riqueza.

Outra explicação de Goldberg para o flerte da população com a direita, é
a covardia. O~indivíduo para ser realmente libertário, tem que ter
coragem de enfrentar todas as fórmulas cômodas de adequação. Ele tem que
se movimentar. O~movimento exige bravura, não braveza.

\begin{center}\asterisc{}\end{center}


\abrefala 

\textbf{Renato Bulcão}: Só para a gente estabelecer uma ligação com a
última vez. O~senhor acredita nos conceitos tradicionais de psicose,
neurose, esquizofrenia? Nessas categorias que foram se estabelecendo
desde o século 19 por todo o século 20?

 

\textbf{Jacob Pinheiro Golberg}: Não acredito. Definitivamente, e cada
vez mais, na minha opinião, classificar comportamentos, formas de
conduta, estilos de vida, maneiras de atuar diante da realidade,
maneiras de perceber ou de sentir dentro de critérios, ou corresponder a
critérios, rígidos são fórmulas simplistas e por isso, extremamente,
perigosas.

 

Nós sabemos que essa é uma tendência que começa na Europa. Começa com a
neurologia, com a psiquiatria, e encontra território fértil,
extremamente fértil, nos Estados Unidos, onde passa a corresponder a
interesses, principalmente da industria farmacêutica. E~a gente poderia
dizer, de certa maneira, a indústria da loucura. Ou a indústria do
desajuste, ou a indústria da repressão. São fórmulas pré"-prontas de
enquadrar o ser humano. O~ser humano não pode ser reduzido, em nenhuma
hipótese, em nenhuma circunstância, a moldes e modelos referenciais.

 

E hoje nós vivemos diante de duas grandes tendências: uma delas que é a
mercadológica, que é exatamente a de fazer esse enquadramento. Também em
psicanálise, psicoterapia, psiquiatria e assim por diante. A~outra é
oposta, aquela que tenta compreender cada pessoa enquanto o seu próprio
mundo. E~nessa hipótese a própria ideia de doença é calibrada segundo o
nível de insatisfação ou de sofrimento da pessoa. Quer dizer, seria a
própria pessoa que poderia se auto"-classificar.

 

Eu me sinto insatisfeito, eu me sinto com penúria psicológica. Desta
maneira, que vai desde uma parede filosófica, até uma parede individual
afetiva, o indivíduo se sente infeliz. Porque ele carrega a infelicidade
do mundo, a ideia do sofrimento do mundo, da vida. A~gente poderia
pensar até numa concepção como a de Schopenhauer, do mundo, até o
indivíduo hipersensível a qualquer privação, a qualquer frustração, a
qualquer dor. Seja uma dor física ou uma dor psicológica de afeto.

 

\textbf{R}: Então pelo que eu entendi, a sua ideia é que qualquer
possibilidade de classificação de qualquer tipo de ``maladia'', digamos
assim, do indivíduo, tem que ser percebida a partir de um processo de
ajuste, que vai do indivíduo à filosofia, ao invés de uma síntese
decorrente de uma teoria geral.

 

\textbf{G}: Você inclusive usou a expressão francesa e os franceses tem
inclusive um conceito do ``maladie d'amour.''(doença do amor). O~Arthur
Koestler tem uma passagem muito interessante, irônica: ele era
psiquiatra, tinha uma formação eclética extraordinária em termos de
erudição.

O Arthur Koestler conta numa passagem em que relata que uma vez ele se
encontrou com um antigo paciente e perguntou: ``Como é que você está?''
E o sujeito relatou: `` Estou sofrendo muito, fui abandonado pela minha
mulher'' E ele falou ``Ah é?! Então a Silvia te abandonou?'' E o paciente
falou: `` Não. imagina, a Silvia é um antigo caso. Depois da Silvia já
teve a Lúcia, a Paulete.'' E o Koestler sarcástico disse: ``Eu passei a
suspeitar que essa ideia do amor romântico na realidade talvez se
estriba num gênero de virose.''

 

Certo tipo de indivíduo é cometido por um vírus ainda não detectado no
laboratório, mas que seria o vírus do amor romântico. Esse indivíduo
está sempre sofrendo num processo masoquista, por alguma perda
romântica. E~isso nós podemos de alguma maneira, ampliar . Têm
indivíduos que estão sempre tristes.

 

Há pouco tempo atrás eu encontrei um antigo amigo meu. Nós começamos a
conversar, e ele virou pra mim e falou: ``Você viu hoje que coisa
incrível a notícia que eu estava vendo aí pela internet?'' E eu falei
``Não…'' Ele estava com um ar catastrófico. E~ai ele disse: ``Um
acidente na Índia matou 18 pessoas.''

 

Claro que a morte de pessoas, em qualquer parte do globo, é triste. Mas
a morte faz parte da fenomenologia do existir. Em qualquer
circunstância, a qualquer momento, os ponteiros do relógio vão apontar
que tem 18 pessoas que morreram por isso ou por aquilo.

 

Então é preciso muito cuidado. A~própria distinção, por exemplo, do
criminoso, do assassino perverso e do psicopata. Fica aí o direito e a
psiquiatria, principalmente a psiquiatria forense, numa discussão que
absolutamente não conduz a nada. A~não ser em alguns estados americanos,
para a câmera de morte. Aqui no Brasil, muito pior que a câmera de
morte, é a tortura permanente nessas prisões que o sadismo do estado
brasileiro cria. É~a oligarquia brasileira atuando, cujo sadismo cria
pequenos campos de concentração.

 

Pequenos em termos só éticos, talvez. Mas nem sei se são éticos
realmente, porque aqui no Brasil eles são muito voltados contra os
negros. Existe também aí um jogo étnico, tal qual havia lá com os
judeus. Mas aí como é que você distingue por exemplo, um diretor de uma
penitenciária nos Estados Unidos, que leva um sujeito a câmara de morte,
e o assassino? Até onde ele é psicopata?

 

Essa pergunta me é sempre formulada, e eu fico perplexo com a ventilação
do problema. Como assim?

 

Um sujeito normal, ele estupra uma criança e mata? Mas como assim? O que
é normal nessa hipótese? Agora se ele não é normal, como ele pode ser
submetido ao processo regular de caráter enquadrável pelo Código Penal,
se ele não é normal?

 

É obvio que a nossa cultura não tem nenhuma resposta lógica para essas
questões que estão além da lógica,fogem da lógica. Elas estão nesses
conteúdos.Os latinos tinham uma expressão interessante para isso: mal
traduzido, ``ardor tremendo''. Quer dizer, essa sensação tremenda que a
gente tem diante do espetáculo do mundo, incompreendido e
incompreensível.

 

\textbf{R}: Dentro dessas perspectivas, do incompreendido e do
incompreensível, o tema é a ideia é do carisma. A~ideia do carisma em
determinados momentos tende para o messianismo, em outros momentos tende
para a destruição de um povo, ou para a maluquice do povo, como se
quiser entender. Mas também pode tender à organização do povo, ou a
organização de valores, que seria uma religião. Dentro de todas essas
possibilidades de uma única qualidade, fale um pouco da ideia do
carisma.

 

\textbf{G}: Desde muito cedo, até por causa da minha formação
filosófica, política, religiosa e assim por diante, eu sempre me
interessei muito pela influência que determinados indivíduos tem sobre o
grupo, sobre a massa, sobre outras pessoas. Na escala de persuasão, na
escala do convencimento, da sedução, da manipulação de opinião pública,
da condução de destino dos grupos e dos povos. Talvez exatamente
motivado por esse interesse, desde muito cedo eu procurei prestar
atenção a esses fenômenos. Desde o colégio eu percebia. Ainda mais que
eu tive o privilégio de estudar em um colégio de missionários
protestantes americanos de Juiz de Fora, o Instituto Granbery.

 

O Instituto Granbery, embora fosse protestante, já sofria um pouco da
influência neo"-pentecostal que posteriormente passou a ser poderosíssima
e maciça no Brasil. A~gente sabe que o neo"-pentecostal, em grande parte
se estabelece em cima dessa ideia carismática. Ideia essa, que acabou
inclusive voltando à sua influência para a Igreja Católica, que não teve
como fugir da questão do carismático.

 

Então a primeira grande personalidade carismática que na infância eu
conheci, foi um missionário norte americano Walter Harvey Moore. Uma
grande figura, extraordinária figura, que teve muito peso na minha
formação. Um sujeito de convicções morais. Um exemplar de solidariedade
humana, e antes e acima de tudo, um condutor de gentes. Ele estava no
interior de Minas, onde havia influência pesadíssima da Igreja Católica,
inclusive da Ação Integralista Brasileira, e ele, embora norte
americano, conseguiu conquistar a simpatia da população da cidade. Ele
era chamado de Mr. Moore,o que hoje se transformou numa legenda, não só
no Brasil mas até nos Estados Unidos.

 

A partir daí, fui conhecendo pouco a pouco, e sempre fiquei muito
motivado, com a relação pessoal indireta. Ainda garoto, eu acompanhei a
visita de Getúlio Vargas a Juiz de Fora. Foi num dos dias que o Estado
Novo costumava celebrar, e isso faz parte da parafernália do entorno do
carisma, e Getúlio sabia explorar isso através do \versal{DIP} (Departamento de
Imprensa e Propaganda). Tinha todo um processo extraordinário que era
chefiado pelo Lourival Fontes, que tinha esse objetivo mesmo. Porque se
você observa, por exemplo, Getúlio Vargas em termos de aparência
pessoal, teoricamente ele seria um indivíduo totalmente despreparado
para a máscara do carisma.

 

Ele era extremamente baixinho, barrigudo , feioso, mas ele sabia
perfeitamente usar os signos que são capazes de servir como
catalizadores da identificação das pessoas. No caso dele, por exemplo, o
charuto. O~charuto é um instrumento óbvio, associado a uma imagética
fálica de poder. Ele não tirava o charuto da boca, numa época de
cigarrinho de palha. O~próprio chapéu que ele usava, um chapéu de
Panamá. Então ele tinha todos os requintes cuidadosos.

 

A oratória do Getúlio tinha a melodia e o cantochão quase monocórdios:
``Trabalhadores do Brasil…'' Tem um caráter, que eu diria que
remete ao hipnótico. O~hipnótico remete ao tantan da selva; o tambor
africano que faz com que as pessoas pensem menos e sintam mais.

 

Daí para frente eu sempre tive essa curiosidade. Olhei o Getúlio, um
pouquinho mais tarde já na adolescência, eu estou falando de alguns
nomes que me vem de imediato, Juscelino Kubitschek quando era prefeito
de Belo Horizonte, e aí, eu já comentei esse caso, eu estou na esquina
da rua Halfeld, e estou conversando com Sagrado Lamir Davi e vejo um
homem descendo a rua. Tem um grupo atrás do homem, e eu viro para o
Lamior e pergunto: `` Quem é aquele sujeito?'' e ele diz: ``Aquele sujeito
é o prefeito de Belo Horizonte, o nome dele é Juscelino Kubitschek''. Eu
virei para o Lamior e disse: ``Ele vai ser Presidente da República''. E~foi! O sinete estava pronto. A~leitura: bastava olhar e perceber; ou ele
seria presidente da república, ou seria presidente da república!

 

Aquele homem ninguém segurava. Se você me perguntar como é isso? É uma
adivinhação? Uma percepção sensorial? Como é que você teve essa
adivinhação? Não tem nada a ver com adivinhação. Tem uma leitura. Eu
acredito que no meu caso é oriunda de toda uma tradição filosófica,
cultural política, religiosa que eu recebi do meu pai e da formação do
colégio. Da minha mãe também e de toda tradição dessa somatória judaica
protestante. Quer dizer, as circunstâncias me ofereceram elementos, que
acabaram sendo instrumentais para essa ordem de análise.

 

Eu não me transformei em psicanalista através da faculdade de
psicologia. Eu sempre fui um autodidata; então eu já era um psicólogo
muito moço. Fui fazer Direito e Serviço Social, porque na época
inclusive não tinha o curso de psicologia. Mas o meu interesse era
compreender a alma humana. Os seus meandros, os seus ritos sinuosos e
principalmente esse, dos jogos de poder através do carisma.

 

Como é que algumas pessoas exercitam isso sobre as outras pessoas? Como
é que elas se submetem. Ainda mais porque os anos que eu vivi, foram
anos em que a personalidade carismática marcou de maneira indelével a
sociologia e a história dos tempos. Para o bem e para o mal.Roosevelt,
Churchill, Stalin,Hitler, Mussolini, quer dizer, era através e em torno
dessas individualidades que as correntes políticas e ideológicas se
degladiavam. Os conflitos eram estabelecidos, também aqui no Brasil.

 

Desde muito cedo isso me fascinou. E~não só no campo da política. Hoje
mesmo eu estava pensando, faz vinte anos que Ayrton Senna morreu. O~Ayrton, que foi uma grande experiência, extraordinária de vida que eu
tive, trabalhar e conviver com o Ayrton É impressionante! Ayrton morto,
ele é escolhido por 46\% da população como o maior ídolo esportivo. Eu
até usei uma frase numa entrevista, na qual eu falo que o Ayrton tinha o
olhar da eternidade. Eu estou convencido disso.

 

Têm certos indivíduos que são imbuídos do seu papel grande no mundo.
Indivíduos que desde muito cedo, ou por nascença, ou por vocação ou, por
esforço, seja pelo que for fadado ao sucesso. Eu poderia desfilar para
você nomes inúmeros da realidade brasileira com os quais, durante esses
anos todos, eu trabalhei, privei, transacionei. Duas oportunidades por
exemplo: com o presidente Lula, uma no sistema Globo de rádio, onde nós
tivemos uma conversa, e a conversa acabou se transformando num debate,
inclusive tenso, veemente, no qual ele falava das origens dele e eu
falava das minhas. Ele do nordeste do Brasil de eu Ostrow, na Polônia.

 

Foi muito curioso, porque algumas das pessoas que estavam perto, que não
tinham o estofo do Lula, eram esses elementos que sempre são os tietes
que acompanham, achando que estavam protegendo o Lula, como se
porventura Lula precisasse ser protegido de uma discussão acirrada, que
era o que ele mais gostava, que é o que ele mais gosta. Aquilo que ele
tem mais prazer. Então nós conversamos, discutimos, uma discussão
bastante veemente, e quando terminou ele me pega pelo braço e faz
questão de descer junto com o filho dele pela Rua das Palmeiras. Fomos
para um bar conversar. E~constatei na conversa sobre judaísmo seu
estranhamento.

 

No que se repetiu depois, quando a Silvia Popovic me convidou pra fazer
a análise psicológica do Lula, diante do Lula, diante de uma câmera de
televisão na \versal{TV} Bandeirantes. Terminado, o Lula levanta e me dá um
beijo. Por que o Lula me deu um beijo? Eu não tinha sido simpático, não
sou nem em geral simpático! Por que ele me deu o beijo? Porque ele é um
sujeito a maior. Ele não tem o entrechoque personalizado, ele não tem o
senso mesquinho da planície. Ele tem o horizonte da montanha.

 

Que é o caso do Jânio. Eu atuei e trabalhei na campanha do Marechal
Lott, outra figura extremamente carismática. Muita gente diz que
Marechal Lott perdeu a eleição porque não era carismático. Mas que coisa
mais reducionista. Eu ouvi inclusive frases do gênero: Lott era um
candidato pesado de ser carregado. Ele não era pesado coisíssima
nenhuma. Não era o momento e as circunstâncias não favoreciam a eleição
dele. Favoreciam a eleição do Jânio Quadros com quem eu tive um debate
no programa da Xênia Bier. Eu estou ali conversando com a Xênia e entra
o Jânio, e eu tinha uma posição antagônica ideológica. Eu tinha feito a
campanha do Marechal Lott e a minha posição era da esquerda
nacionalista. Em questão de minutos eu fui conquistado, seduzido pelo
charme do Jânio.

 

Mas que charme era esse? Era um charme de professor de Vila Maria, que é
um ``uomo qualunque''; um homem qualquer. Como Jango era um homem
qualquer e é um paradoxo, como Ayrton Senna era um homem qualquer?

 

\textbf{R}: Qual é a base deste paradoxo? Como pode um homem qualquer se
transformar em sedutor, pessoa que move, seduz.

 

\textbf{G}: Engraçado, você falou seduz e eu acabei ouvindo Jesus. Eu
acho que talvez não exista um exemplo mais extraordinário de um homem
comum, na minha opinião, me desculpem os religiosos, ou melhor ainda, me
desculpe aqueles que fazem parte de religião, porque os religiosos vão
entender o que eu estou dizendo: Mesmo para quem acredita piamente na
condição messiânica de que Jesus era um messias, ele há de compreender
que Deus elege exatamente um homem comum.

Porque este é um contexto teológico da ideia do Messias, de Jesus. Mais
do que ninguém, o tempo todo, ele dá a demonstração de simplicidade. se
aproximando dos pobres. É~difícil imaginar o quanto.

Você imagina que só o presidente do Citibank de Nova Yorque pretende ser
carismático? Jamais! Não existe isso. Em Nova Yorque, o que pode existir
na minha opinião, isso sim, é um sujeito como Roosevelt, paraplégico,
que se transforma num sujeito irresistível. E~quem você queira. São
sempre indivíduos que se identificam com aquilo que cada um de nós tem
de mais humano.

 

Voltando aí com Senna era exatamente isso: Ayrton Senna do Brasil! De
certa maneira, é como se ele se multiplicasse através de todos.

 

\textbf{R}: Quer dizer então que essa ponte é feita, como você falou no
início de Getúlio, de signos. Da utilização de signos, na medida em que
a pessoa não tem medo de se apropriar dos signos e de recriar esses
signos de uma forma que eles façam sentido para o homem comum, que eles
também são. Você acaba tendo a possibilidade de identificação muito
grande com essa pessoa, porque você entende o que ele está falando e ele
de alguma maneira não te põe medo. Ele cria algum tipo de mimésis entre
a sua opinião e a opinião que ele está expressando?

 

\textbf{G}: Você usou aí um raciocínio que eu acho que facilita muito a
compreensão do carisma. Você falou; ``…não tem medo''. Se existe
algo comum, que eu vi por exemplo num jogador de futebol,com o Sócrates,
com o Giba, o jogador de vôlei, o Brizola, enfim, com várias dessas
pessoas. O~que sempre chamou muita atenção e e eu dou muita importância
nessa questão é a expressão do corpo. No entendimento do carisma, é o
olhar. Se você presta atenção, é um olhar que passa através de você. Mas
passa através de você porque? Eu agora nesse momento estou falando com
você, me lembrando de um instante. Eu estou tomando café da manhã com o
Brizola. O~Brizola desce pelo elevador e estende a mão para mim. Aquilo
me impressionou muito, porque seu olhar coincide com o olhar do Ayrton
Senna. E íamos sair do Hotel Maksoud para o velório Senna, em seguida.

 

O tempo todo a impressão é assim: que o olhar desse tipo de indivíduo, é
o olhar que leva em conta você, que leva extremamente em conta o outro.
Ninguém é desimportante para carismático! Mas ele não se limita em você.
Ele passa através de você. Tanto é que a expressão ``visionário'' na
minha opinião, ela é muito aparentada com conceito do carismático. O~quixotesco, ele é capaz de viver também no outro mundo.

 

\textbf{R}: O quixotesco de alguma forma se fecha em si mesmo.O juízo de
realidade dele não é compartilhado com mais ninguém.

 

\textbf{G}: Não é! Mas de qualquer maneira você vê que tem algo comum
entre o quixotesco e em geral, os carismáticos. Porque dificilmente o
carismático tem a completude em vida, do concreto. Em geral fica um gap
entre o que ele pretende e seu grande sonho.

 

Nós estávamos falando de político e assim por diante. Por exemplo o
Jango, a respeito de quem eu estou acabando de escrever um ensaio, com
uma análise psicológica: João Goulart. Muita gente empresta à Jango
deméritos injustos.

Desde alegando que ele era não era um estadista, que ele era um político
menor , um oportunista que cresceu a sombra de Getúlio Vargas. E~finalmente que era um covarde, por não ter resistido à investida
totalitária e integralista fascistoide do Mourão Filho, este sim um
paspalho.

 

É muito interessante porque às vezes o indivíduo se pretende um herói, e
na realidade, em termos de história, não passa de um palhaço. Tanto é
que ele mesmo se intitulava ``uma vaca fardada''. De um lado a gente tem
então uma vaca fardada, e de outro lado, na minha opinião, esse sim, um
herói capaz de renunciar a uma resistência, que teria levado o Brasil a
um massacre absurdo.

 

O Jango na minha opinião tinha isso. É~muito interessante porque antes,
eu tinha falado do olhar. O~Jango não olhava nos olhos de ninguém. Era
uma das características do Jango. Ele tinha aquele andar de marinheiro,
por causa da perna, problema que ele tinha por causa do aleijão, e ele
manquejava. Mas o Jango, na minha opinião, é a figura central daquela
década. Ele foi expurgado. Ele não é levado em conta. Como se
praticamente ele nao existisse. Como se ele não tivesse tido importância
nenhuma e por que isso?

 

É muito curioso, eu não tenho dúvida nenhuma, que um pouco mais cedo, um
pouco mais tarde, vai haver o resgate da figura dele. Porque essas
figuras não ficam no limbo da história. Elas acabam emergindo exatamente
pelo carisma.De alguma maneira vai passar.

 

\textbf{R}: Na verdade, a esquerda também contribui bastante para esse
esquecimento de Jango, porque foi o homem que teve medo de ser um herói.

 

\textbf{G}: No sentido talvez da frase extraordinária do Bertold Brecht:
``Pobres dos povos que precisam de heróis!''. Eu acho muito importante
distinguir nessa questão toda a coragem autêntica, generosa, do Jango no
discurso histórico sindicato, que muitos acham que foi o pretexto para o
golpe militar. Ele foi corajoso, ele só teve a autenticidade de dizer do
que o Brasil precisava, como aliás continua precisando, tantos anos
depois, de reforma agrária, mudança no sistema de ensino, discussão do
papel das forças armadas. As questões todas que ele colocou continuam em
pauta e ainda não foram resolvidas.

 

\textbf{R}: Dentro da idéia do carisma, a política então seria uma
atuação mediada por pessoas da mais diversas qualidades carismáticas,
mas também por pessoas das mais diversas mediocridades. Nesse sentido,
tentando resgatar o início da nossa conversa, sobre o que o psicótico,
neurótico, nós sabemos que no meio dessas milhares de pessoas que fazem
política, certamente a psicologia comum enquadraria fulano como tal,
ciclano como tal e beltrano como tal.

 

O que faz com que todas essas pessoas, independente agora do seu carisma
ou da sua mediocridade, se lancem numa tentativa conquistar o espaço
público? Hoje em dia, mas também já desde 1960, de um espaço
midiatizado, e de alguma maneira uma tentativa de idolatria de si mesmo?

 

\textbf{G}: Eu acho que a política tem o poder que a gente poderia
chamar `` libidinoso'', o poder de gozo,de satisfação, da intensidade e
da exuberância de viver. Através da política, mais do que talvez
qualquer outra atividade humana, você vê a possibilidade de mudança, nos
jogos do real, da sociedade. Isto na minha opinião tem duas grandes
veredas: uma delas é o interesse pessoal e narcísico. Aí você vê o
corrupto, o canalha, o medíocre, o infame.

 

Nós sabemos que grande parte da história política do Brasil se faz mais
nas páginas policiais dos jornais, do que realmente nas páginas de
reflexão. Aliás são pouquíssimas na nossa mídia. Cada vez mais a mídia é
usada como instrumento desses interesses mesquinhos de grupos de
controle.

 

Mas de outro lado, sem dúvida nenhuma você encontra essas pessoas
vocacionadas para o bem comum. Simplesmente o indivíduo que tem a
virtude, no conceito latino do indivíduo virtuoso. O~indivíduo que
sacrifica inclusive interesses pessoais, em nome do bem público, em nome
do outro.

 

Se em outras épocas da história essas personalidades --- Joana D`Arc-
essas personalidades se ofereciam através da religião e da fé, hoje
muitos se dedicam sim, à política. O~grande problema sempre é fazer a
distinção verdadeira, independente de questão ideológica, daqueles que
tem essa vocação, e aqueles outros que até usam os seus poderes
pessoais, os seus recursos humanos, para corresponder à interesses
menores.

 

Na verdade são traidores da esperança daquilo que o ser humano tem de
mais deslumbrante, que é a crença no outro. Quando você trai a crença
que o outro deposita em você, é pérfido.

 

\textbf{R}:Vou fazer a última pergunta: É pérfido, mas aparentemente
agora no século 21, a esquerda é sempre a tentativa daqueles que
acreditam no outro, ou na possibilidade de convivência e cooperação com
o outro. A~direita é basicamente aquela que não acredita no outro. É~aquele que imagina que ele é melhor que o outro e poderia hierarquizar o
processo liderado por ele mesmo. E~isso poderia trazer benefícios
pessoais. Aparentemente é mais ou menos isso, pois esquematicamente está
assim dividido. Nesse sentido, por que a maior parte das pessoas, namora
mais com a direita do que com a esquerda, mesmo sabendo a escolha da
esquerda é que iria indiretamente beneficiá"-lo?

 

\textbf{G}: Na minha perspectiva, eu não vejo a esquerda como uma
realidade homogênea. Eu acho que existem muitas esquerdas, e uma
direita. Existe uma direita, na minha opinião, que de alguma forma
pretende ser a frenagem da história.

 

Pretende que a sociedade permaneça infantilizada, enquanto uma casta
possa deter o poder. Seja o poder da cultura, da academia militar,
social,financeiro ou religioso e assim por diante. Em todas as atitudes
e todos os níveis.

 

Na esquerda existem várias esquerdas. E~existe principalmente dentro da
esquerda, matizes que na realidade são profundamente de direita. Não
vamos esquecer que Hitler chefiava um partido que se chamava Nazional
Sozialistische Deutsche Arbeiter Partei, Partido Nacional dos
Trabalhadores Alemães. Ele alegava que era nacional --- socialista,
chegando ao extremo de usar as músicas que eram músicas compostas pela
esquerda, eram hinos revolucionários, e eles passaram a usar para a
propaganda do Goebbels.

 

Então é preciso muito cuidado com essa ideia. Agora, feita essa
depuração, a minha posição, é público, é anarco"-socialista. A~minha
posição é da liberdade individual extrema, absolutamente extrema. Eu
costumo dizer e repetir : não existe liberdade mais ou menos. Ou é
livre, ou não é.

 

O ser humano não tem outra alternativa: nós nascemos e tendemos para a
liberdade. O~socialismo tem que ser uma forma de distribuição de
riqueza, só e única e exclusivamente. No momento em que se transforma em
aparelho de poder, está se aproximando da direita e do fascismo. Isso
aconteceu principalmente na Guerra Civil Espanhola, que foi um
microcosmo de todos esse conflitos. E~esta é acima de tudo uma das
tragédias da esquerda, que talvez explique isso que você perguntou.

Por que a população acaba tendo um flerte com a direita? É uma das
explicações. A~outra explicação, sem dúvida nenhuma, é a covardia. O~indivíduo para ser realmente libertário, ele tem que ter coragem de
enfrentar todas as fórmulas cômodas de adequação. Ele tem que se
movimentar. O~movimento exige bravura, não braveza.\footnote{Esta entrevista pode ser assistida na íntegra em~goo.gl/o\versal{QK}k5Z.} 

 

\fechafala 
