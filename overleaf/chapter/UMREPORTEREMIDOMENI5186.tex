\chapterspecial{Um repórter em Idomeni}{}{Sérgio Lírio}
 

O leitor deve estar cansado de ouvir sobre a suposta crise terminal do
jornalismo. Diante da vasta e caleidoscópica oferta na internet e da
profunda mudança cultural no consumo e na produção de informação, que
eliminou a barreira entre emissores e receptores, há cada vez menos
cidadãos dispostos a pagar por notícias.

A revolução tecnológica não é, porém, a única responsável pelo declínio
dos meios tradicionais de comunicação. Empresas e jornalistas, de uma
maneira transtornada e compulsiva, cavaram a vala que cada vez mais os
enterra na irrelevância. Outrora pilar da democracia, o jornalismo,
salvo raras exceções, deixou"-se embriagar pela burocracia, pela defesa
dos interesses do poder, transitório ou real, pelo desprezo às
necessidades dos leitores e pela recusa diária em cumprir seu papel de
serviço público. Trocou o difícil, mas consagrador exercício de ``gastar
as solas dos sapatos'' e testemunhar os acontecimentos pelo conforto dos
preconceitos refrescados pelo ar"-condicionado.

No centro dessa decadência está o que alguns convencionaram chamar de
``a morte da reportagem''. Embora cause um certo efeito na plateia, não
chega a ser um veredicto correto. Os bons repórteres, de fato, foram
banidos das redações convencionais, mas não são poucos aqueles que, pelo
amor à profissão, encontraram novas formas de levar adiante o
compromisso de contar histórias, simplesmente porque alguém precisa
conta"-las.

É o caso deste livro assinado por Gabriel Bonis. Como se constatará em
cada página de ``Refugiados de Idomeni'', o autor se equilibra entre o
auto de fé do jornalista e a racionalidade do pesquisador. Entre a
eloquência necessária para relatar um dos principais dramas
contemporâneos e o cuidado para não se deixar levar pelas inevitáveis
emoções.

Bonis reúne essas duas características. É~um repórter impetuoso,
clássico, e um acadêmico rigoroso, mestre em Relações Internacionais
pela Queen Mary, Universidade de Londres, especializado em políticas de
asilo. Lembro do dia em que o conheci. Ainda estudante de jornalismo,
Bonis insistiu até que eu o recebesse na redação. Estava decidido a
conseguir um emprego em CartaCapital. A~insistência acabou recompensada.
Em pouco tempo, revelou uma maturidade incomum, sólida formação e
talento raro. Não me surpreendeu a decisão de trocar o Brasil por uma
temporada de estudos no exterior e muito menos a escolha de aceitar o
desafio de uma \versal{ONG} inglesa para atuar no campo de refugiados de Idomeni,
pequena aldeia grega na fronteira com a Macedônia.

O livro é um relato cru e preciso do cotidiano de homens, mulheres e
crianças confinados em uma espécie de limbo, divididos entre as
lembranças dos horrores da guerra, da fome e da desagregação e a
esperança de sobrevivência e de reencontro. Espremidos pela crueldade de
quem os expulsou de casa e a intolerância de quem não aceita abrigá"-los
em ``sua terra''. Depauperados, famintos, abandonados, tragicamente
humanos.

Esta obra renova as esperanças de quem se recusa a aceitar o ocaso do
jornalismo. Como uma fênix, ele renasce cada vez que alguém se propõe a
dividir uma história com os leitores. E~ressurge vigoroso, inegavelmente
essencial para a compreensão do mundo a nossa volta. E~não é essa a
razão da própria existência: Entender, antes de tudo?
