
% https://www.flickr.com/photos/cafodphotolibrary/22092745854

\putodd{}

\imagemgrande{Foto: Ben White/CAFOD, outubro de 2015. \ccby}{./img/FlickCC.jpg}\clearpage

\chapterspecial{Em alto mar}{}{}
 
% Sugestão de Subtítulo: A trajetória em alto mar
% * <bbonis@gmail.com> 2017-01-27T13:34:39.705Z:
% 
% > A trajetória em alto mar
% Pode ser.
% 
% ^.
Eram duas horas de uma madrugada de fevereiro de 2016 quando aquele bote
inflável de borracha partiu da costa da Turquia levando 45 pessoas a
bordo. Quinze delas eram crianças. A~escuridão serviria como camuflagem,
escondendo a embarcação das autoridades turcas no trajeto rumo à Grécia
pelo Mar Egeu. A~força das ondas castigava a frágil embarcação com um
chacoalhar intenso, despertando em Rami\footnote{ O nome do refugiado não foi revelado por completo
para proteger sua identidade.}  algo além da ânsia a subir por seu estômago. O~pavor sentido por ele provavelmente
também tomava outros passageiros  daquele ``barco''.  


Eram duas horas da madrugada quando o bote inflável partiu da costa da Turquia levando 45 pessoas a
bordo. Quinze delas eram crianças. A~escuridão serviria como camuflagem,
escondendo a embarcação de borracha das autoridades turcas no trajeto rumo à Grécia
pelo Mar Egeu. Era fevereiro de 2016. A~força das ondas castigava intensamente a frágil embarcação, despertando em Rami\footnote{ O nome do refugiado não foi revelado por completo
para proteger sua identidade.}  algo além da ânsia que subia por seu estômago. O~pavor sentido por ele provavelmente
também tomava outros passageiros  do ``barco''. 


``Tudo o que se via é água. Não se avistava ninguém por horas'', contou o
sírio de 18 anos.

Boiando sob o chacoalhar das ondas, na sombra da noite, Rami
acreditou, contudo, ter visto uma pessoa ``dormindo no mar'', a poucos
metros de distância do bote. Não seria uma cena surpreendente em uma
rota na qual o mar se transformou em cemitério para milhares de
refugiados.

``A viagem de barco é muito perigosa'', ele enfatizou.

E os riscos são ainda maiores para os passageiros de embarcações não
projetadas para longos percursos em alto"-mar. Naquele bote, algumas
mulheres e crianças choravam assustadas, enquanto outros viajantes
tentavam encontrar uma posição confortável. Amontoados e tensos, todos
vestiam jaquetas flutuantes, mas o acessório não era o bastante para que
se sentissem seguros.

``O estado do bote era muito ruim'', lembrou Rami.

Os números ilustram a dimensão dos problemas daquela jornada: de acordo
com a Organização Internacional de Migração (\versal{OIM}), 5,143 pessoas
morreram ou desapareceram tentando chegar a solo europeu pelo mar em
2016, contra 3777 em 2015\footnote{ Organização Internacional de Migração (2017)
\emph{Missing migrants project}. Disponível em:
{goo.gl/wF40Lo}
(Acesso: 17 de Junho de 2017).}. Pelas rotas em direção ao
litoral da Grécia, 856,723 mil refugiados entraram na Europa em 2015, de
um total de 1 milhão que chegou ao continente pelo Mediterrâneo e pelo
Mar Egeu\footnote{ \versal{ACNUR} (2017) \emph{\versal{UNHCR} refugees/migrants emergency
response --- Mediterranean}. Disponível em:
goo.gl/31yJ59
% data.unhcr.org/mediterranean/regional.php 
(Acesso: 08 de Janeiro
de 2017).}. Em 2016, as novas entradas somaram 361,709
mil pessoas, das quais mais de 173,447 mil chegaram pelo litoral grego,
segundo o \versal{ACNUR}, a agência da \versal{ONU} para refugiados\footnote{ Ibid}. A~maioria dos refugiados veio do Afeganistão,  do Iraque e da Síria, que enfrenta
uma guerra civil desde 2011.

O bote que trazia Rami entrou em águas europeias sem ser interceptado,
evitando que seus passageiros fossem enviados de volta à Turquia. Até
então, aquela embarcação, que poderia naufragar tão facilmente caso
perfurada, havia resistido à força das ondas. Foi uma pequena
vitória, embora não houvesse motivos para comemorar. A~segurança nas
terras do Velho Continente ainda não estava garantida, conforme ficou
claro quando o ``barco'' enfrentou, algum tempo depois de atingir a
metade do caminho, talvez o seu maior desafio para evitar a deriva.

``O mar estava violento e o bote seguia em linha reta. De repente, ele
começou a rodar'', descreveu Rami, e continuou,``Foram cinco giros. Pensei que o
bote fosse afundar''.

Os gritos de desespero, disse, tomaram conta dos passageiros. ``Estava assustado
porque tinha meus primos comigo. Minha mãe chorava'', contou.
``Achei que iria morrer. Então pedi para que minha namorada se cuidasse
caso isso acontecesse.''

O bote não afundou naquele momento porque o ``capitão'' conseguiu
retomar o controle do motor, mas ele sucumbiria poucas horas depois
quando a água invadiu o "barco" através das perfurações no assoalho de plástico.

Às sete da manhã, após cinco horas em alto"-mar, o bote começara a
afundar no sul da Grécia. A~marinha grega evitou dezenas de mortes ao
resgatar os passageiros e levá"-los para Mytilini, a capital da ilha de
Lesbos, um famoso destino de turistas europeus. Começava ali a jornada
daqueles 45 indivíduos na Europa.

Em Lesbos, e nas demais ilhas onde os botes chegassem, as autoridades
eram obrigadas por regulações da \versal{UE} a registrar os refugiados e a
capturar suas impressões digitais, que seriam compartilhadas com outros
países do bloco por um banco de dados comum. Depois do procedimento,
eles poderiam seguir de balsa para a capital Atenas.

Rami, sua mãe, o tio e os dois primos (dois garotos de quatro e seis
anos) precisavam chegar o quanto antes a Idomeni,no norte da Grécia e fronteira com a Antiga República Iugoslava da Macedônia. Aquele era o primeiro país da chamada ``rota dos Bálcãs'', um trajeto pelo qual refugiados cruzavam também Sérvia, Croácia, Eslovênia e Áustria para chegarem à Alemanha, destino almejado pela maior parte deles.

A~rota mais curta, contudo, havia sido interrompida em outubro de 2015 quando a
Hungria terminou de erguer cercas em suas divisas com a Sérvia e a
Croácia para impedir a entrada de refugiados em seu
território\footnote{ Anistia Internacional (2015) \emph{Hungary: \versal{EU} must
formally warn Hungary over refugee crisis violations}. Disponível em:
goo.gl/\versal{STE}ca4
(Acesso: 12 de outubro de 2016).}.

A decisão húngara e os diversos indícios de que Alemanha e Suécia
estavam à beira de atingir os limites de suas capacidades para receber
solicitantes de proteção internacional forneciam fortes 
evidências de que a rota dos Bálcãs seria fechada a qualquer momento.
Ainda assim, Rami e sua família decidiram parar na cidade de Kozani, no
norte grego, por cinco dias. A~escolha se provou equivocada pouco
tempo depois.

A porta de entrada dos refugiados para os países mais desenvolvidos da Europa era um
vilarejo tranquilo e isolado, a 86 quilômetros de distância de Salônica,
a segunda maior cidade da Grécia. Idomeni é um povoado rural onde,
segundo o censo de 2011\footnote{ Press Office of the Spokesperson for the Management
of the Refugee Crisis, (2016). \emph{Request for comments on Idomeni and
refugees in Greece --- Brazilian Book}. {[}email{]}.} , vivem 154 pessoas que
cultivam pequenas plantações de berinjelas, melancias, tomates e
repolhos, além de uma extensa produção de milho, espalhada por longos
lotes de terra.

O lugarejo contrasta com os 2,3 milhões de habitantes que Aleppo, a
maior cidade da Síria e antigo centro econômico"-industrial do país,
chegou a ter em 2005. Para escapar da guerra naquela metrópole, Rami
percorreu 1,8 mil quilômetros até Idomeni, onde uma simples cerca de
arame farpado interrompeu a jornada de um jovem que havia sobrevivido a
estilhaços de bombas em uma zona de conflito e à travessia do Mar Egeu.

``Quando chegamos em Idomeni, não nos deixaram seguir viagem'', contou
o sírio.

Em 9 de março de 2016, a República da Macedônia fechou completamente a sua
fronteira com a Grécia para refugiados, deixando milhares de
pessoas presas em um dos piores acampamentos de refugiados da Europa.
Aquele era o último suspiro da rota dos Bálcãs, cimentada dois dias
antes pelo polêmico acordo da \versal{UE} com a Turquia para que o país receba de volta todos ``os migrantes que não necessitem de proteção
internacional'' chegando ao litoral grego.
% PARA O AUTOR: o acordo persiste? se sim, "para que o país receba...", se não, "para que o país recebesse de volta todos os migrantes que não necessitassem..."
% * <bbonis@gmail.com> 2017-01-27T13:43:01.285Z:
% 
% > % PARA O AUTOR: o acordo persiste? se sim, "para que o país receba...", se não, "para que o país recebesse de volta todos os migrantes que não necessitassem..."
% 
% O acordo ainda é válido. Mudei lá o tempo verbal. 
% 
% ^ <anaclara@hedra.com.br> 2017-01-27T16:17:26.177Z:
%
% ok
%
% ^.

Ancara também aceitou receber todos os indivíduos interceptados em suas águas e concordou com a
atuação da \versal{OTAN} no Mar Egeu e em apertar o cerco contra o tráfico de
pessoas\footnote{ European Council --- Council of the European Union
(2016) \emph{\versal{EU}"-Turkey statement, 18 march 2016}. Disponível em:
% consilium.europa.eu/en/press/press"-releases/2016/03/18-eu"-turkey"-statement/
goo.gl/wtkMTx
(Acesso: 30 de novembro de 2016).} .

% Sugestão de Subtítulo: A vida difícil no vilarejo
% * <bbonis@gmail.com> 2017-01-27T13:43:41.485Z:
% 
% > % Sugestão de Subtítulo: A vida difícil no vilarejo
% Mesmo comentário anterior sobre subtítulos dentro de capítulos. Acho que o leitor não vai se perder na história. 
% 
% ^.
Em 17 de março, o acordo foi estendido para que todos os ``migrantes
irregulares'' que chegassem às ilhas gregas a partir de 20 de março de
2016 e não solicitassem asilo nelas fossem retornados à
Turquia.\footnote{ Ibid.}  Ficou definido ainda que, para cada sírio
enviado das ilhas gregas para a Turquia, um outro seria reassentado na
\versal{UE} até um número máximo de 72 mil. A~Turquia receberá do bloco 6 bilhões
de euros até 2018 para lidar com os refugiados\footnote{ European Commission (2016b) \emph{\versal{EU} and turkey agree
European response to refugee crisis --- European commission}. Disponível
em: goo.gl/e2naq4
% ec.europa.eu/news/2016/03/20160319\_en.htm 
(Acesso: 30 de
novembro de 2016).}.

``A vida em Idomeni é como o inferno'', destacou Rami. ``Se você deseja
morrer, pode ir para lá'', prosseguiu.

O local foi comparado, inclusive, a campos de concentração nazistas pelo próprio
ministro do Interior grego, Panagiotis Kouroublis, que não hesitou em
chamá"-lo de Dachau moderno\footnote{ Campo de concentração na Alemanha nazista.  goo.gl/FkJbyu }
 ``um resultado da
lógica das fronteiras fechadas''. ``Qualquer um que vier aqui sentirá
vários socos no estômago'', disse ele após uma visita em 18 de março.
Uma declaração que o governo grego classificou como
``infeliz''\footnote{ Press Office. \emph{Request for comments on
Idomeni}\emph{.} {[}email{]}.} , mas que não o foi.

Naquele momento, as condições de vida em Idomeni eram descritas como
``abismais'' e ``insuportáveis'' por \versal{ACNUR}\footnote{ \versal{ACNUR} (2001) \emph{Greece: \versal{UNHCR} concerned at
conditions in new refugee sites and urges that alternatives be found}.
Disponível em:
goo.gl/Ac7wv2
%unhcr.org/news/briefing/2016/5/57480cb89/greece"-unhcr"-concerned"-conditions"-new"-refugee"-sites"-urges"-alternatives.html
(Acesso: 12 de outubro 2016).}  e
pela \versal{ONG} Anistia International\footnote{ Anistia Internacional (2016b) \emph{Refugees
shamefully trapped in Greece --- amnesty urgent actions}. Disponível em:
goo.gl/BMa3Ak
%ua.amnesty.ch/urgent"-actions/2016/03/048--16/048--16--1?ua\_language=en
(Acesso: 12 de outubro de 2016).} ,
respectivamente. A~situação degradava"-se com rapidez. A~chuva quase
constante de fevereiro e março transformara em um imenso lamaçal os
campos, até então utilizados para plantações, mas que àquela altura
haviam sido tomados por uma imensidão de barracas de acampamento a
abrigar milhares de pessoas. 

Com as mãos e a ajuda de um pedaço de madeira, uma mulher cavava uma vala  
ao redor de sua pequena tenda. Era assim que os refugiados tentavam evitar
que seus abrigos tão precários alagassem. A água seguiria o caminho cravado
no chão, escorrendo para longe das tendas. Mas isso não bastava para mantê-las
secas. 

``Nossa barraca ficava molhada porque chovia demais. E~não havia espaço
para todo mundo nela. Era muito frio e tive que passar
três dias dormindo do lado de fora'', Rami contou.

Pelas ruelas de terra batida e sem nome, roupas e cobertores abandonados 
acumulavam"-se em grande número. Dividiam espaço com garrafas plásticas,
sapatos e outros dejetos que transbordavam de algumas caçambas. No chão,
um rastro, dentre outras coisas, de fraldas sujas e restos de alimentos.
O vento contante, ao menos, dissipassava o forte cheiro de urina e podridão.

Algumas dessas ruas foram criadas para receber um centro de transição 
erguido em setembro de 2015 pela organização francesa Médicos Sem Fronteiras
(\versal{MSF}). Parte delas surgiu do aterro de áreas antes destinadas a plantações. 

O enceramento da rota dos Bálcãs marcou a terceira vez em que as portas do
norte da Europa fecharam"-se para os refugiados em Idomeni. Após a
República da Macedônia, a Croácia e a Eslovênia selarem definitivamente suas
fronteiras para os refugiados vindos da Grécia, o centro de transição
construído para abrigar temporariamente cerca de 1,5 mil pessoas
entrou em colapso. Em março de 2016, o local havia se transformado em um campo
não oficial de refugiados, com o \versal{MSF}\footnote{ \versal{MSF} International. 2016.
\emph{{\#Greece}}\emph{:
Around 14,000 migrants 15 refugees are currently trapped in}
\emph{{\#Idomeni}}\emph{.
We did over 2,000 medical consultations in one week} \emph{in
×}\emph{\versal{MSF} {[}Twitter{]}.} 8 de março. Disponível em:
{goo.gl/jrXqQM}
{[}Acesso: 12 outubro de 2016{]}.}  e o
Conselho Dinamarquês para Refugiados\footnote{ Conselho Dinamarquês para Refugiados (2016)
\emph{Summary of regional migration trends middle east (march, 2016)}.
Disponível em:
goo.gl/ZFM9V0
(Acesso: 12 de outubro de 2016).}  (Danish
Refugee Council, em inglês) estimando sua população em 14 mil pessoas. O~governo grego, entretanto, contestou esses números e reconheceu que
``apenas'' 8,5 mil estavam no centro no momento da sua evacuação final,
em 26 de maio\footnote{ Press Office. \emph{Request for comments on
Idomeni}\emph{.} {[}email{]}.}.

Quando chegou a Idomeni, no começo de março, Rami passou a integrar essa
superpopulação que se aglomerava a poucos metros das cercas que a separava 
do território macedoniano, em uma área de aproximadamente 11 campos de futebol
com as dimensões do Maracanã. Ali, milhares de pessoas enfrentavam 
temperaturas congelantes e ventos intensos na esperança de que as 
fronteiras voltassem a se abrir em algum momento.

Naquela pequena cidade de barracas de plástico forjada rapidamente, a vida  
seguia como se exemplificada no improviso de cabos pendurados entre uma tenda 
e outra, ou entre postes de eletricidade, para servir como varal de roupas. 
As cercas no entorno da linha do trem que corta Idomeni também enchiam"-se de 
trajes e cobertores molhados aguardando uma fresta de sol no céu quase sempre
nublado e chuvoso.

Nem todas as nuvens cinzentas sob Idomeni eram naturais.

Massas de fumaça negra circulavam sobre o centro de transição
dia e noite. O~ar pesado, quase sempre carregado de fuligem, era o
efeito mais evidente da falta de abrigo adequado durante o outono e o inverno 
europeu. Do lado de fora das poucas tendas do \versal{MSF} equipadas com 
sistemas de aquecimento, os refugiados queimavam tudo o que pudessem encontrar
para manterem"-se aquecidos. Vasculhavam as caçambas de lixo em busca de 
caixas de papelão ou as pediam a funcionários de organizações humanitárias. 
Muitos cortaram árvores nas redondezas do vilarejo para alimentarem suas fogueiras. 
À~maioria que não conseguia esses materiais, restava apenas queimar 
roupas, calçados e garrafas de plástico para evitar a hipotermia.

Tossir era quase sempre inevitável. Por vezes, o ar continha um odor tóxico como se viesse do escapamento 
de um carro com o motor à beira do colapso. Outras vezes, cheirava a pneus incendiados em algum protesto. 
Esses cheiros agarravam-se às roupas, mesmo após serem lavadas diversas vezes.
% * <bbonis@gmail.com> 2017-01-27T13:50:48.550Z:
% 
% > Essas partículas do ar caíam nas roupas, impregnado os tecidos mesmo após a lavagem.
% 
% Mudei aquela frase. Vejam se está melhor agora, por favor. 
% 
% ^ <anaclara@hedra.com.br> 2017-01-27T15:09:34.067Z:
%
% bem melhor!
%
% ^.
% "aroma de churrasco" talvez devesse mudar ou elimiar essa última parte do parágrafo

% Sugestão de Subtítulo: O fim do sonho 
% * <bbonis@gmail.com> 2017-01-27T13:51:09.850Z:
% 
% > % Sugestão de Subtítulo: O fim do sonho 
% remover
% 
% ^.
Aos poucos, o sonho de cruzar a fronteira tão próxima foi desbotando-se 
com as evidências de que a Europa buscava construir ainda mais muros - 
físicos e políticos - para dificultar a chegada de refugiados ao continente. 
Isso havia ficado claro poucas semanas antes da chegada de Rami a Idomeni, 
quando as autoridades da República da Macedônia impuseram um rígido controle
fronteiriço. Passaram a exigir passaportes dos refugiados e a impedir a entrada
de sírios não provenientes de áreas em intenso conflito. Damasco, por exemplo,
era considerada minimamente segura. Logo, refugiados vindos da capital síria
não eram, em geral, autorizados a acessar o território macedoniano via Idomeni.

Rapidamente, o número de pessoas liberadas no vilarejo grego para seguir viagem
pela rota dos Bálcãs caiu para apenas algumas centenas por dia. A~fila de sírios
aguardando na fronteira aumentou, como consequência. Ao menos, parte
deles ainda poderia cruzar para a República da Macedônia, ao contrário
de afegãos e iraquianos que, dias antes, passaram a não ser mais
considerados como indivíduos em situação de emergência humanitária por
países do norte europeu. 

A~espera era, contudo, angustiante e longa. Tão longa que a fronteira fechou
por completo sem que Rami a cruzasse. Mas lá ele ficou com sua familia
por um mês pelo ``sonho de ver a fronteira se abrir novamente''.

O fim da rota dos Bálcãs criou um drama humanitário de largas proporções
em Idomeni. O~pequeno vilarejo tornou"-se um limbo político de onde
refugiados pareciam não ter como escapar. ``A situação aqui é trágica
\redondo{[…]} Não honra o mundo civilizado, não honra a Europa'',
afirmou Dimitris Avramopoulos, Comissário para a Migração, Assuntos
Internos e Cidadania na Comissão da \versal{UE}, em mais uma das visitas oficiais
de autoridades europeias ao local\footnote{ European Commission (2016) \emph{Commissioner
Avramopoulos in Idomeni}. Disponível em:
goo.gl/Mpe88n (Acesso: 15
de outubro de 2016).}.

O contraste entre o discurso humanitário de representantes da esfera
política do continente e os esforços reais da \versal{UE} para solucionar o
problema era abismal. Por meses, o centro de transição funcionou à base
de geradores que frequentemente sucumbiam. A água potável em alguns
pontos do campo vinha de caixas d'água instaladas por atores
humanitários. O~\versal{MSF} precisou convencer as autoridades locais a fornecer
ao menos água e eletricidade para facilitar as operações das \versal{ONG}s.

As condições sanitárias nos acampamentos ao redor da fronteira eram péssimas. 
Com os banheiros químicos sempre imundos, não era incomum encontrar fezes ou 
urina pelo caminho. Muitos refugiados eram forçados a lavar suas roupas nas 
mesmas torneiras em que buscavam água para beber.

``Tudo lá era imundo. Ficamos sem banho por dez dias porque nunca havia
água quente no chuveiro'', disse Rami. ``Não podíamos tomar banho gelado
porque ficaríamos doentes'', explicou. A~família precisou adotar
medidas extremas para manter as crianças saudáveis:``Cortei os cabelos
dos meus primos com medo de que algum inseto começasse a viver na cabeça
deles'', revelou.

\imagemgrande{O lixo acumulava-se nas caçambas e no chão rapidamente.}{./img/DSC_0098.jpg}

O governo grego nunca reconheceu o centro de transição como oficial.
Eximiu"-se da responsabilidade de melhorar as condições de vida do local,
alegando não ter a intenção ``de apoiar a criação de um centro de
acomodação para refugiados em uma área de fronteira'' que, segundo
Atenas, não possuía condições para tal, mas ``foi obrigado a tolerar a
existência daquele acampamento'' até que estivesse em uma posição de
transferir seus residentes para abrigos ``mais
adequados''\footnote{ Press Office. \emph{Request for comments on
Idomeni}\emph{.} {[}email{]}.}.

Foi em um destes abrigos ``mais adequados'' em Oreokastro, um dos
centros de recepção oficiais próximos de Salônica, que Rami contou
sua história. Ele é um jovem alto, magro, de cabelo escuro com as
laterais raspadas bem curtas. Naquele dia, vestia uma regata branca 
que deixava à mostra tatuagens em um dos braços. A barba fina não 
escondia no rosto as marcas da adolescência.

Rami falava com desenvoltura em um inglês bem compreensível. Enquanto
caminhava pelas ruelas do abrigo - um galpão escuro convertido às pressas 
para alojar 1,5 mil refugiados em tendas padronizadas - eventualmente ele 
parava para conversar com algum vizinho e, certa vez, pediu um cigarro a um 
conhecido. Sem conseguir o que queria, seguiu andando em direção a um portão 
de ferro. Era a entrada daquela antiga fábrica, e de onde chegava a única fonte 
de luz natural do local naquele momento. Apresentou seus colegas, que aos poucos 
se dispersaram quando ele começou a falar sobre Aleppo.


Na metrópole síria, a vida era bem diferente daquela na tenda que dividia
com a mãe, o tio, vítima de uma doença degenerativa, cuja evolução já o 
impedia de lembrar"-se de quem era, e os dois primos, tão miúdos que poderiam ser
carregados sem esforço com apenas um braço. A~realidade em que se
encontravam na Europa não fazia parte do imaginário de nenhum deles
quando deixaram seu país de origem em janeiro de 2016.

Em Aleppo, a família morava em uma casa confortável. Tinha uma vida
``boa'', como disse Rami. Entretanto, em mais uma noite dos seis anos de guerra
civil que destruíram o país, uma bomba atingiu a residência enquanto todos
dormiam.

``Meu pai morreu enquanto o segurava em meus braços'', disse Rami. Seu
olhar parecia congelado.

Ele não tinha ideia de quem havia lançado a bomba, mas naquele período
Aleppo era dividida entre grupos rebeldes e as tropas do presidente
Bashar al"-Assad, que desde 2013 usava as forças governamentais para
realizar ataques aéreos devastadores na cidade. Aquela estratégia
permitiu o avanço governista às custas de milhares de vidas de civis.
A partir de outubro de 2016, Assad intensificou os bombardeios à
cidade com o apoio militar da Rússia e de milícias patrocinadas pelo
Irã, além de impor um cerco a Aleppo\footnote{ The Guardian (2016) \emph{Aleppo must be `cleaned',
declares Assad, amid outcry over bloody siege}. Disponível em:
goo.gl/hW2m4m
(Acesso: 15 de outubro de 2016).}. Em
novembro, as tropas do governo conquistaram áreas que antes estavam sob o poder
rebeldes.

No fim de dezembro, o regime retomou o controle de Aleppo e, após um
acordo entre as partes do conflito, dezenas de milhares de civis e
também rebeldes puderam ser evacuados da cidade com a ajuda de
organizações internacionais como o Comitê Internacional da Cruz
Vermelha\footnote{ \versal{BBC} (2016) \emph{What}\emph{'}\emph{s happening in
Aleppo?} Disponível em:
goo.gl/lNt0Pi
%bbc.com/news/world"-middle"-east-38132163 
(Acesso: 8 de Janeiro
de 2017) e \versal{BBC} (2016a) \emph{Aleppo battle: Syrian city}
\emph{`}\emph{back under government control}\emph{'}. Disponível em:
goo.gl/bc3H98
%bbc.com/news/world"-middle"-east-38408548 
(Acesso: 8 de Janeiro
de 2017).}.

Rami deixou Aleppo meses antes da derrota dos rebeldes. Ainda na Síria,
no entanto, ele entrou para uma estatística comum: juntou"-se aos milhares
de civis forçados a abandonar os estudos devido à falta de segurança.
Ele cursava o primeiro ano da faculdade de Medicina. Na Europa, queria
retomar os estudos para se tornar um médico generalista.

``Era o sonho do meu pai'', revelou, com a voz embargada. ``Antes de morrer, 
ele me pediu para não abandonar os estudos, para começar de novo.''

Em Oreokastro, aquele parecia um sonho distante. Rami estava à salvo,
mas ainda não havia conseguido recuperar o controle de sua vida.
