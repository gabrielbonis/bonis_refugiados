\chapterspecial{O Respeito pelo Paciente}{}{Renato Bulcão (\emph{org.})}
 

Conhecia Jacob Pinheiro Goldberg das entrevistas de televisão. Sua falta
de sorrisos e de falas mansas davam a impressão de um terapeuta muito
sério, guardião da moralidade. Fui procurá"-lo por indicação de uma
amiga. Contei meu caso e avisei --- não tenho um tostão agora. Ele me
acolheu, me atendeu, me pôs em terapia de grupo, e não desistiu de mim
por dois anos. Um dia disse que eu estava resistindo em avançar. E~que
nossas conversas tinham virado uma perda de tempo.

Fui viver a vida e me descobri mais apto do que antes. Voltei curioso em
entender o que tinha acontecido. Que técnica era aquela? Psicólogos
amigos meus diziam que era lacaniana. Mas psicólogos lacanianos
insistiam que não era. De fato, as três características lacanianas
utilizadas por Goldberg eram àquela altura, a adoção do tempo lógico
(terminamos por hoje), atender o telefone no meio da sessão (a vida não
para…) e uma sala de espera lotada obrigando muitas vezes uma
intimidade incômoda entre os pacientes.

No mais, tudo diferia das experiências anteriores que eu tinha
experimentado desde os 16 anos como paciente. Principalmente aquele
silêncio sepulcral de muitos terapeutas, que me dava a impressão de um
taxímetro correndo enquanto estamos parados há horas num engarrafamento
no caminho para o aeroporto, num dia de chuva, numa cidade estrangeira,
e o motorista não entende a sua língua.

Gravei vários vídeos com ele, mas ele sempre resistiu em falar de
técnica. Sua trajetória acadêmica foi original e polêmica. Portanto,
quem estuda Goldberg na academia, estuda seus poemas e sua literatura.
Essa inexistência do psicólogo didaticamente apresentado me incomodava.
Seu sucesso precisa ser compartilhado, pelo menos com meus alunos de
filosofia do curso de psicologia.

De tanto insistir, conseguimos chegar a um acordo em 2015. Fizemos cinco
diálogos que registramos em vídeo. Fiz a transcrição, e depois corrigi
um pouco o texto para ficar legível. Para a publicação, argumentei ainda
que eram necessárias explicações simplificadas para as novas gerações.
Conheço meus alunos. Muitos não fazem ideia que Getúlio Vargas se matou,
nem quem foi João Goulart.

Transcrevemos para manter o vigor da oralidade, sua preocupação
estilística. O~título inicial de trabalho sugerido por ele foi
``Psicanálise heterodoxa'', e tem raízes históricas. Na história da
psicologia, temos Georg Groddeck e Donald Woods Winnicott que admitem em
comum que corpo e psiquismo são dimensões de uma mesma realidade e não
duas essências distintas; concebem o adoecimento como via de regresso a
estágios anteriores do desenvolvimento e veem a natureza como força viva
e direcionada para a saúde. São psicólogos heterodoxos. Em Belo
Horizonte encontramos o médico e psicanalista Gregório Baremblitt,
médico e psicanalista argentino, que amadureceu sua obra no Brasil. Na
Argentina foi membro fundador do grupo psicanalítico denominado
Plataforma, e no Rio de Janeiro e em São Paulo, o Instituto Brasileiro
de Psicanálise, Grupos e Instituições (Ibrapsi), e ainda o Instituto
Félix Guattari de Belo Horizonte. As heterodoxias, me lembrou o
psicanalista uruguaio Valentin Guerreros, sempre propõem uma proximidade
com o contexto, que produz o real para o paciente.

Em todas as heterodoxias da psicanálise, encontramos componentes da
sócio"-análise, a percepção do contexto em que o sujeito existe na
sociedade. A~sócio"-análise tem uma variação interessante proposta por
Deleuze e Guattari, chamada esquizo"-análise. No caso de Goldberg, sua
inspiração direta advém de David Cooper e seu movimento de
anti"-psiquiatria.

Aliás, a característica de permitir a existência do real me atraiu muito
no processo de Goldberg. Nas suas sessões de grupo ficava claro que
ninguém sofre sozinho, e que os sofrimentos --- aflições na sua
nomenclatura --- não diferem tanto assim de uma pessoa para outra,
independente de gênero, etnia, idade ou condição social. Nas encenações
de sua dramaturgia terapêutica, havia o cuidado em reproduzir o contexto
incluindo a realidade social de cada um.

Espero que no futuro jovens psicólogos adotem algumas de suas
percepções. Até porque o contrato de Goldberg com os pacientes é muito
eficaz no que diz respeito à compreensão do processo de tratamento e
alívio. Nunca vou me esquecer de uma amiga minha, ex"-paciente, que
declarou : Já entendi que ele tem razão, mas ainda não estou pronta para
casar e ter filhos. Interrompeu a terapia. Anos depois, casou e teve um
filho.

Há quem o acuse de moralismo. Posso afirmar que neste sentido a posição
de Goldberg é puro Voltaire: Discordo totalmente do que diz, mas defendo
até a morte o teu direito de dizê"-lo. Esse respeito pela liberdade do
outro é muito difícil de ser encontrado em psicólogos. A~maioria não
abre mão de seus juízos.

Espero que este livro seja de leitura rápida aos leitores, mas que as
ideias de Jacob Pinheiro Goldberg fiquem por muito tempo ecoando em seus
pensamentos.

\smallskip{} 

\hfill {}Dezembro de 2015

\vfill

\emph{Nota Bene}: Os diálogos a seguir foram transcritos na íntegra das
gravações de vídeo disponíveis no You Tube. Eventuais correções foram
feitas para facilitar a leitura. O conteúdo e seu vigor
permanecem intactos.

\part{Uma visão da psicanálise} 
