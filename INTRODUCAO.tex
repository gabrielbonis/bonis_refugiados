\chapterspecial{O que significa ser um refugiado?}{}{Erika Sallum}


Conflitos e perseguições têm forçado mais gente a deixar suas casas do
que em qualquer outra época desde que a \versal{ACNUR}, a agência das Nações
Unidas para refugiados, começou a coletar informações sobre o tema, nos
anos 1950. Dados do fim de 2015 mostraram que, pela primeira vez na
história da \versal{ONU}, o número de ``deslocados'' ultrapassou os 60 milhões.
Mais precisamente, existem hoje cerca de 65,3 milhões de pessoas
``deslocadas'' de suas casas à procura de mais segurança -- isso
equivale, em média, a 24 seres humanos fugindo por minuto, quatro vezes
mais que na década anterior. Dentre esses 65,3 milhões, cerca de 21,3
milhões enquadram-se na definição internacional de ``refugiados'', dos
quais metade tem menos de 18 anos (sendo 98.400 desacompanhados dos pais
ou parentes).

As estatísticas assustadoras não param por aí: ainda segundo a agência,
até 2015, a cada 113 pessoas no planeta, uma é refugiada, deslocada
interna ou solicitante de asilo. Se contabilizarmos o número de
refugiados, deslocados internos e solicitantes de asilo, temos uma
população maior que a de países como a França. Nada menos que 53\% dos
refugiados existentes no mundo hoje vêm de apenas três países: Somália,
Afeganistão e Síria. E, diferentemente do que muita gente é levada a
crer devido à divulgação de fotos e notícias sobre pessoas tentando
chegar à Europa, a vasta maioria dos refugiados não se encontra no
continente: 85\% dos que estão sob o mandato da \versal{UNHCR} vivem em países em
desenvolvimento -- a exemplo do Líbano, onde há um refugiado a cada 5
habitantes, colocando-o na liderança das nações que mais abrigam esse
grupo em relação a sua população\footnote{Edwards, Adrian. \emph{Global forced
  displacement hits record high}.
  {unhcr.org/} (20 de
  junho de 2016){]} e \emph{\versal{ACNUR}: Deslocamento forçado atinge recorde
  global e afeta 65,3 milhões de pessoas;}
  nacoesunidas.org; 20
  de junho de 2016.}. 



Como já declarou o coreano Ban Ki-Moon,
ex-secretário-geral da \versal{ONU}, ``os números são desconcertantes. Cada um
deles representa uma vida humana. Mas não se trata de uma crise de
números. É uma crise de solidariedade''\footnote{\emph{Chefes da \versal{ONU}
  pedem mais solidariedade com as pessoas forçadas a se deslocar;}
  {nacoesunidas.org/}; 7
  de outubro de 2016.}.



Curiosamente, por outro lado, nunca estivemos tão conscientes sobre a
existência dessas pessoas, já classificadas de ``umas das mais
vulneráveis'' do planeta pelo português António Guterres, ex-alto
comissário das Nações Unidas para os refugiados e atual secretário-geral
da \versal{ONU}\footnote{\versal{ACNUR}, \emph{Global Trends -- Forced Displacement in
  2015}.}.  Nunca houve tantos dados e pesquisas, de órgãos
governamentais e \versal{ONG}s, envolvendo indivíduos em fuga tentando proteger a
própria vida. Poucas vezes a mídia do mundo todo deu tanto espaço para o
martírio de sírios, afegãos, africanos e muitos outros que se arriscam
na tentativa de chegar à Europa. Como diz a expressão popular, ``o
elefante está na sala'', e a ele não se pode mais ignorar -- ainda mais
na era das redes sociais, das notícias postadas em velocidade recorde,
das fotos de celular que tudo captam.

Mas o que exatamente significa ser um refugiado? Quem deve protegê-lo
quando seu próprio Estado torna-se o algoz ou incapaz de fazê-lo? Qual a
diferença entre refugiado, deslocado interno e imigrante (aliás, isso
importa mesmo?)? Quando a comunidade internacional se omite e vira as
costas para uma das maiores e delicadas crises do mundo contemporâneo, a
quem essas pessoas podem recorrer?

Como já afirmou a própria \versal{ACNUR}, a prática de conceder asilo em terras
estrangeiras a pessoas que estão fugindo de perseguição é ``uma das
características mais antigas da civilização, e há referências a isso em
textos escritos há mais de 3.500 anos''\footnote{\versal{ACNUR},
  \emph{Deslocando-se através das fronteiras}.
  {goo.gl/149XIt};
  2016.}. Entretanto foi só após a
Primeira e a Segunda Guerra Mundial que a comunidade internacional, em
conjunto, decidiu se unir para desenvolver documentos que estabelecessem
alguns princípios comuns sobre o tema, em uma tentativa de, pelo menos
em teoria, amenizar o problema. De acordo com os documentos da \versal{ONU}:

\textit{Refugiado} é toda pessoa fugindo de conflitos armados, violência
generalizada e violações graves de direitos humanos em seu país. Sua
situação é tão frágil e insuportável que ela se vê obrigada a cruzar as
fronteiras de seu Estado para buscar abrigo em outra nação. E é
exatamente isso que a define como refugiada: devido a um \emph{fundado
temor} de perseguição por motivos de raça, nacionalidade, opinião
política ou participação em grupos sociais, essas pessoas não podem (ou
não querem) voltar para casa. Ao chegar a outro país, o refugiado deve
pedir asilo às autoridades locais. Até o final de 2015, segundo a \versal{ONU},
cerca de 3,2 milhões de pessoas eram \textbf{solicitantes de asilo}
(outro número recorde)\footnote{\emph{Tendências Globais sobre
  refugiados e outras populações de interesse do \versal{ACNUR}};
  goo.gl/OtMEzV}.


Quando não se chega a ultrapassar as fronteiras e se permanece dentro do
território nacional, então o termo muda: trata-se de uma pessoa
\textit{deslocada interna} (tradução meio estranha do termo em inglês
\emph{internally displaced people}). Segundo a \versal{ONU}, existem 40,8 milhões
de deslocados internos no mundo (dados do fim de 2015), o maior número
já registrado na história\footnote{Ibid.}. A grande parte dos deslocados
internos vive na Colômbia, Síria e Iraque. E, ainda que se sintam
perseguidos e ameaçados, estão sob a jurisdição de seus governos -- e
outros países não podem interferir sem o consentimento das autoridades
locais, o que deixa o deslocado interno em uma situação de extrema
fragilidade.

A grande parte dos deslocados internos
vive na Colômbia (essa população foi criada ao longo da guerra civil do
país) , Síria e Iraque. E, ainda que se sintam perseguidos e ameaçados,
estão sob a jurisdição de seus governos e da \versal{ACNUR} -- e outros países
não podem acessá-los sem o consentimento das autoridades locais, o que
deixa o deslocado interno em uma situação de extrema fragilidade. Com
6,6 milhões de deslocados internos desde o início da guerra, a Síria é
um triste exemplo dos desafios que a ajuda humanitária têm enfrentado
para chegar aos civis -- o que levou, em janeiro de 2016, à divulgação
de um apelo internacional assinado por mais de 120 organizações
não-governamentais e agências da \versal{ONU} para que o acesso à população seja
possível e imediato.
% talvez nota sete.

Já \textit{migrantes} são aqueles que se mudam (de região ou país) não
em razão de perseguições ou ameaças, mas principalmente para melhorar de
vida por meio de, por exemplo, um novo trabalho, estudos ou relações
pessoais. Assim sendo, saem voluntariamente de seus países e para eles
podem voltar quando quiserem. Cada país possui suas próprias leis em
relação a imigrantes, porém quando a pessoa em fuga se enquadra nas
definições de refugiada os governos signatários das convenções relativas
ao tema têm a obrigação de respeitar não apenas a legislação interna
como também as normas internacionais referentes ao assunto. Por isso é
essencial distinguir um imigrante de um refugiado assim que ele chega a
uma nova fronteira; e por isso tantos países europeus vêm incorretamente
apontando os milhares de refugiados em situação de fuga e perigo que têm
aparecido em suas fronteiras como uma ``crise de migração econômica'',
em uma tentativa de se eximir das responsabilidades adquiridas quando
assinaram os documentos internacionais. Dados da \versal{ACNUR} são claros em
mostrar que a imensa maioria dos passageiros dos barcos que têm tentado
cruzar o Mar Mediterrâneo nos últimos anos é composta por sírios,
afegãos, iraquianos e nacionais de outros países em conflito, e que não
podem retornar a suas casas e devem receber proteção internacional\footnote{Refugees/Migrants Response -- 
Mediterranean: goo.gl/MQ6I6B}.




Esses documentos internacionais chancelados pela \versal{ONU} englobam,
basicamente, dois principais acordos sobre refugiados: a Convenção
Relativa ao Estatuto dos Refugiados, de 1951; e o Protocolo Relativo ao
Estatuto dos Refugiados, de 1967.

A convenção foi criada no período do pós-Guerra, quando os países
arrasados pelos conflitos que destruíram boa parte da Europa e outras
nações como Japão decidiram se reunir para desenvolver melhor mecanismos
comuns que impedissem o recrudescimento da violência em seus territórios
e defendessem os direitos de quem não pode mais contar com a proteção do
próprio Estado. Sua base provém de antigos documentos internacionais
sobre refugiados e, principalmente, do artigo 14 da Declaração Universal
dos Direitos Humanos, proclamada pela \versal{ONU} em 10 de dezembro de 1948, que
diz que ``toda pessoa, vítima de perseguição, tem o direito de procurar
e de gozar de asilo em outros países''. No segundo parágrafo do artigo,
a Carta determina que ``esse direito não pode ser invocado em caso de
perseguição legitimamente motivada por crimes de direito comum ou por
atos contrários aos propósitos e princípios das Nações Unidas'' -- ou
seja, criminosos fugitivos não se enquadram na definição de refugiados.

Ratificada por 145 países, a Convenção adotada em 1951 (e que entrou em
vigor em 1954) ainda é o principal documento internacional em defesa do
refugiado hoje. Além de definir o termo ``refugiado'', descreve os
direitos dos deslocados e aponta algumas condutas que os Estados
deveriam tomar (apesar de fazer referência ao refúgio, os dois
documentos não tratam claramente da concessão de refúgio). Um dos
princípios mais importantes da Convenção de 1951 é o do
\emph{non-refoulement} (não devolução), que afirma que nenhum refugiado
deve ser deportado ao seu país natal caso esteja correndo risco de morte
ou sofrendo ameaças contra suas liberdades individuais. Segundo a \versal{ACNUR},
a Convenção deve ser aplicada sem discriminação por raça, religião, sexo
ou país de origem\footnote{Acnur, \emph{O que é a
  Convenção de 1951?,}
  goo.gl/S80EG1.}.



Pensada para incluir pessoas que se tornaram refugiadas em decorrência
de conflitos ocorridos antes de 1º de janeiro de 1951 (em uma tentativa
de limitar a ajuda aos afetados apenas pela Segunda Guerra), a Convenção
de 1951 precisou ser ampliada quando a comunidade internacional não pôde
mais virar os olhos para o surgimento de novas ondas de deslocados mundo
afora. Surgiu assim o Protocolo de 1967, no qual os países signatários
devem aplicar o que está no texto de 1951, agora sem limites de datas ou
localização geográfica. E, de acordo com esses documentos, é a \versal{ACNUR} a
responsável por fiscalizar os direitos dos refugiados, ainda que o órgão
não possua poder supranacional e tenha de respeitar e trabalhar em
parceria com os governos de cada país. Cabe aos Estados reconhecer
legalmente um refugiado, mas a agência da \versal{ONU} conduz esse processo em
diversos países que não possuem condições ou não desejam realizar o
procedimento por conta própria.

O Protocolo de 1967 surgiu para acudir uma nova onda de crises de
refugiados pós-Segunda Guerra, dentro e fora do continente europeu. Os
movimentos de descolonização da África nos anos 1950 e 1960,
impulsionados pela queda do poder da Europa, levaram milhões de pessoas
a se deslocarem. A separação do Paquistão da Índia, no fim dos anos
1940, provocou uma avalanche de gente atravessando as novas fronteiras
traçadas na Ásia. Guerras de independência obrigaram outras levas de
humanos a abandonarem seus lares -- só a Guerra da Biafra, que assolou
a Nigéria em 1967, deslocou 2 milhões de pessoas dentro e fora do país.

Acompanhado a expansão dessas crises, a \versal{ACNUR} passou a atuar no mundo
todo, e não mais apenas na Europa: hoje a agência está presente em 128
países, possui 10.700 integrantes e viu seu orçamento anual inicial de
\versal{US}\$,300 mil, em 1950, saltar para \versal{US}\$,5,3 bilhões, em 2013\footnote{
	\versal{UNHCR}: Figures at glance; 
	goo.gl/yaTtsf}.

Organizações não-governamentais também tomaram para si parte desse papel
e hoje se tornaram atores importantes, a exemplo da Refugees
International, do Comitê Internacional da Cruz Vermelha, do
International Rescue Committee, entre outras.

Com o aparecimento de outros conflitos -- em especial o elevado número
de guerras civis desde os anos 1990 que estouraram com o fim da Guerra
Fria -- e, por consequência, outras ondas de deslocados, a \versal{ACNUR} e os
países integrantes das \versal{ONU} vêm enfrentando novos desafios, em especial
em relação à determinação de quem se enquadra na definição de refugiado.
O termo \emph{fundado temor} (de ser perseguido por motivos de raça,
religião, nacionalidade ou opinião política...) envolve elementos
objetivos, porém também subjetivos, por isso exige que cada caso seja
analisado individualmente. O que é um temor para uma pessoa pode não ser
para outra. Além disso, não existe uma definição universal e exata para,
por exemplo, a palavra ``perseguição''.

Como aponta a \versal{ACNUR}, ``devido às variações dos perfis psicológicos dos
indivíduos e às circunstâncias de cada caso, as interpretações sobre o
conceito de perseguição podem variar''\footnote{Acnur,
  \emph{Manual de procedimentos e critérios para a determinação da
  condição de refugiado}.}. Assim, as autoridades que
analisam o pedido de asilo devem entender o contexto geográfico, social
e político, além da história pessoal de cada indivíduo. E, apesar de a
Convenção de 1951 e o Protocolo de 1967 não se referirem explicitamente
em proteger pessoas fugindo de guerras ou torturas, outros documentos
(como a Convenção contra Tortura de 1984) e o direito internacional
começaram a incluir esses casos para ajudar a definir se uma pessoa é ou
não refugiada. A questão da orientação sexual também passou a integrar
as discussões internacionais: uma pessoa perseguida em seu país por ser
\versal{LGBTI} (``I'' aqui englobando o termo ``intersexo'') pode ser
classificada de refugiada? Segundo diretrizes dos Princípios de
Yogyakarta, adotados por um painel de especialistas em 2007, ``toda
pessoa tem o direito de buscar e de desfrutar de asilo em outros países
para escapar de perseguição, inclusive de perseguição relacionada à
orientação sexual ou identidade de gênero''\footnote{Ibid.}  
-- e, ainda que esse
documento não seja vinculante, ele reflete princípios já consolidados do
direito internacional. Há inúmeros casos de indivíduos homossexuais que
receberam o status de refugiado em países europeus porque em seus países
de origem o homossexualismo é um crime passível de pena de morte.

São muitos os desafios da comunidade internacional diante das
incessantes crises de deslocamento forçado. Conflitos geradores de
refugiados como os da Síria, e outros que não ganham tanto as páginas
dos jornais (como os Tamils que fugiram do Sri Lanka ou a perseguição da
minora étnica muçulmana dos rohingya em Myanmar), colocam à prova a real
capacidade de os países ajudarem quem bate à porta porque já não pode
mais voltar para sua casa. Como afirmou Filippo Grandi, atual alto
comissário da \versal{ONU} para os refugiados em um discurso para o European
Policy Center, em dezembro de 2016, ``no front dos refugiados, nós
precisamos de ações imediatas e robustas para evitar uma `corrida para o
abismo' na qual os países renunciem a medidas comuns ao passarem a
acreditar que apenas soluções nacionais funcionem''\footnote{Grandi,
  Filippo. \emph{Protecting refugees in Europe and beyond: Can the \versal{EU}
  rise to the challenge?}, 5 de dezembro de 2016.}. A hora de agir
é agora. Daí a importância crucial de livros como este, que traz um
testemunho comovente de quem tem visto de perto uma das piores e mais
vergonhosas crises da humanidade.

