\chapterspecial{Psicanalise da Espera}{}{}
 

 

\letra{E}{m geral}, todos nós entendemos ``a espera'' na nossa vida como um
``não"-tempo'', como uma ``não"-existência'', é como se fosse uma pausa na
nossa história, entre aquilo que já aconteceu e aquilo que a gente está
na expectativa de que possa vir a ocorrer. É~como se a gente fizesse uma
suspensão no tempo.

O que é mesmo muito parecido com o que as pessoas presumem que seja o
sono. Ou seja, o sono seria uma temporada entre o dia e a noite. Uma
temporada essa que é quase como se você estivesse congelado e o fluir da
tua história estivesse interrompido. Evidentemente se trata de uma
armadilha psicológica grave, porque nós vivemos durante a espera e
durante o sono. Durante o sono a gente sonha. Na espera nós fantasiamos
o que vai acontecer, nós recordamos o passado e nós apostamos no futuro.
Então a espera pode ser muito mais do que a ocorrência, o clímax das
nossas emoções.

No momento que você me informou que estava fotografando o entorno visual
aonde você se situava aguardando a sessão de terapia, eu achei
extremamente original, porque você, diferentemente do que muitos
psicanalistas, inclusive Freud e Lacan, que achavam que a sessão de
análise começava quando terminava, você tinha capturado que ela, na
realidade, tinha a continuidade estabelecida no ``sempre'', e a espera
era já um dos momentos importantíssimos que deveriam ser registrados. E~você fazia o registro de que maneira? Através do seu instrumento de
trabalho, através da percepção sensorial que você tem da vida, que é o
olhar, a fotografia e o cinema. E~isso que você tem feito, na minha
opinião, deveria ser transmitido, porque deixa de ser seu, como toda
obra de arte, e passa a ser de todo mundo, que em qualquer
circunstância, na vida, está esperando. Mas indo mais longe,
transcendendo, e isto é uma posição particularmente minha, eu acho que
nós sempre vivemos a espera, porque eu sou o filho de uma tradição
messiânica, eu estou sempre na expectativa da utopia, eu estou sempre
atrás do paraiso. Aliás, nada mais impressionante, na natureza, do que o
tempo gravídico, que é um tempo de espera.

\begin{flushright}~\end{flushright}

